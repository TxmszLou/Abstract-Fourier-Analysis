
\section{Compact Groups}

% In previous sections, we have shown the existence and the uniqueness of
% isotypic decomposition of representations of finite groups.  However, the major
% theorem we would like to prove was about the similar decomposition but for
% general compact groups.

% In the following section, we will introduce notions of compact groups, and
% properties of representations on them: can they be decomposed, is the
% decomposition is isotypic?
% explore the classical development of
% representation theory of compact groups.

% \begin{defn}[Topology]
%   A collection $\tau$ of subsets of set $X$ is said to be a \emph{topology} of
%   $X$ if it satisfies:
%   \begin{enumerate}
%     \item[(i)] $\varnothing \in \tau$, $X \in \tau$
%     \item[(ii)] If $V_i \in \tau$ for $i = 1, \ldots , n$, then $\bigcap_{i = 1}^{n} V_i \in \tau$
%     \item[(iii)] If $\{V_{\alpha}\}_{\alpha \in I}$ in an arbitrary collection
%       of subsets of $\tau$, then $\bigcup_{\alpha \in I} V_{\alpha} \in \tau$.
%   \end{enumerate}

%   If $\tau$ is a topology of $X$, then $X$ is called a \emph{topological space},
%   whose open sets are elements in $\tau$.  $X$ is \emph{locally compact} if
%   every point in $X$ has a compact neighborhood.

%   A function $f : X \to Y$ between two topological spaces are said to be
%   \emph{continuous} if $f^{-1}(V)$ is open for every $V \subset Y$ open. $f$ is
%   said to be \emph{continuous at $x_0$} if to every neighborhood (open set) $V
%   \ni f(x_0)$, there exist a corresponding neighborhood $W \ni x_0$ such that
%   $f(W) \subset V$.
% \end{defn}

We would like to answer the similar question about complete reducibility on
general compact groups.  Is it possible to develop similar results on finite
reducibility for general compact group?  Is it possible for us to adapt proofs
in the finite case to compact groups?

We will provide a affirmative answer to the first question: we will be able to
show the reducibility of representation of a compact group into a orthonormal
set of irreducible representations, each of them is finite dimensional.
However, unfortunately we cannot adapt most of the proofs in the finite case to
the compact case (in fact, the only theorem we developed previously may be
carried over to the compact case is Schur's Lemma and complete reducibility of
\emph{finite} representations).
% We will need to develop a quite large
% toolbox of machineries that allow us to establish the result for arbitrary
% representations.

In short, the orthogonality will be established by Schur's orthogonality
theorem, and the complete reducibility will be established by Peter-Weyl's
Theorem. 


% \begin{exa}
%   Consider group
%   \[
%     G = \SetForm{\begin{pmatrix} y & x\\ 0 & 1 \end{pmatrix}}{x,y \in \R, y > 0}
%   \]
%   It has left- and right-invariant measures:
%   \[
%     d\mu_L = y^{-2} dx dy,
%     \quad
%     d\mu_R = y^{-1} dx dy
%   \]
% \end{exa}

\subsection{Complete Reducibility for Finite Representations}

Let $G$ be a compact group, similar to the finite case, we can define the notion
of a representation of $G$:  at the most general setting, a
\emph{representation} of $G$ is simply a group homomorphism $f : G \to H$, where
the group $H$ is not even necessarily associated with a linear structure.
However, in this paper, we will be only interested in \emph{linear
  representations}, thus $H = \End(V)$ for some vector space $V$.  However, in
contrast with the finite case, here $V$ may not have a finite basis.  When $V$
is infinite dimensional and is associated with a topology, we say $(\pi,V)$ is
\emph{irreducible} if it has no proper nonzero invariant \emph{closed}
subspaces.

In the following sections, we will be interested in general of $V$ being a
Hilbert space, that is, an inner product space which is complete under the norm
induced by the inner product. In particular, we will be interested in the
Hilbert space $L^2(G)$ of square integrable functions on $G$. The inner product
is given by conjugate integration (this is why we developed the theory of
integration on compact groups with Haar measures earlier):
\begin{equation} \label{Hilbert-inner-product}
  \langle f_1 , f_2 \rangle_{L^2} \triangleq \int_G f_1(g) \conj{f_2(g)} dg
\end{equation}
In addition, $L^2(G)$ is indeed a complete metric space with the metric induced
by the inner product defined above, for a detailed examination, see Theorem 3.11
in \cite{rudin1987real}.

In fact, similar to the case of finite group, in the general case of $G$ being a
compact group, we can show the existence of an $G$-equivariant inner product.
This also means, every finite dimensional representation of a compact group is
\emph{unitarisable}.

\begin{thm}
  Let $G$ be a compact group and $(\pi,V)$ be a finite dimensional
  representation of $G$, then $V$ admits a $G$-equivariant inner product.
\end{thm}

\begin{proof}
  We will again apply Weyl's trick of averaging over the group $G$, but this
  time by integrating over $G$: let $H_0(,)$ be an arbitrary inter
  product on $G$, define
  \[
    H(v,w) \triangleq \int_G H_0(\pi(g)v, \pi(g)w) dg
  \]
  Then, $H(,)$ is $G$-equivariant by construction.
\end{proof}

If we replace ``sum over group finite group $G$'' by ``integrate over compact
group $G$'' in \Cref{ortho-inv} and \Cref{fin-decompose}, etc. The result of
complete reducibility follows for \emph{finite} representations of compact group
$G$.

% \begin{thm}
%   Let $(\rho,V)$ be a finite dimensional unitary representation of compact group
%   $G$, $W$ is an invariant subspace of $V$. Then $W^{\perp}$ is also an
%   invariant subspace of $V$.
% \end{thm}

% \begin{thm}
%   Let $G$ be a compact group, $(\rho,V)$ be a finite dimensional representation
%   of $G$, $W$ be a subrepresentation of $V$.  Then there exist a complementary
%   invariant subspace $W'$ of $V$ such that $V = W \oplus W'$.
% \end{thm}

\begin{thm}[Complete reducibility for finite representations]
  Any finite representation of a compact group $G$ admits an orthogonal
  decomposition into irreducible sub-representations.
\end{thm}

Now, let's step to the complete reducibility of arbitrary representations.

\subsection{Matrix Coefficients}

Let $G$ be a compact group, $(\pi,V)$ be a representation of $G$.  Then if $V$
is finite dimensional, by choosing a basis for $V$, we can write out the matrix
representation of the linear map $\pi(g)$ for every $g \in G$.  Or equivalently,
we have a collection of functions $\phi_{ij} : G \to \C$ that tells us what is
the value of the $(i,j)$th matrix entry of $\pi(g)$.  Even more importantly, if
we are able to construct such a collection of functions, then we also obtain a
representation of $G$! The question is, how do we construct this collection of
functions?

Let's brute force through one possible construction. Under the given basis $B =
\{e_i\}_{i = 1}^{n}$, suppose we have a vector $v = (v_1, \ldots, v_n)^{T} =
\sum_1^n v_i e_i$ and an element $g \in G$. Suppose the matrix $\pi(g)$ is
represented by:
\[
  \begin{pmatrix}
    \phi_{11}(g) & \cdots & \phi_{1n}(g)\\
    \vdots & \ddots & \vdots\\
    \phi_{n1}(g) & \cdots & \phi_{nn}(g)
  \end{pmatrix}
\]

Computing $\phi(g)(v)$ gives, $\pi(g)(v) = (\sum_{1}^{n} \phi_{1i}(g) \cdot v_i,
\cdots, \sum_{1}^{n} \phi_{ni}(g) \cdot v_i)^{T}$, hence $\pi_{ij}(g) =
L_i(\pi(g)e_j)$, where $L_i$ is a linear functional on $V$ that picks out the
$i$-th component of the vector, $L_i(\sum a_j e_j) = a_i$.

This motivates us to define \emph{matrix coefficients}, an
abstract characterization of those family of functions independent of choices of
a basis or even a concrete vector space $V$.

\begin{defn}[Matrix Coefficients]
  Let $G$ be a group, $(\pi, V)$ be a finite dimensional representation of $G$,
  then \emph{matrix coefficients} on $G$ are maps of the form
  \[
    \phi : g \mapsto L(\pi(g)v)
  \]
  for $L \in V^*$, $v \in V$.
\end{defn}

Let $\mathcal{M}_{\pi}$ denote the set of matrix coefficients of representation
$(\pi,V)$, then it forms a vector space. Furthermore, we can show an intimate
relation between matrix coefficients and the dual representation.

\begin{prop} \label{dual-mat-coeff}
  $f$ is a matrix coefficient of $(\pi,V)$ iff $\check{f}(g) \triangleq
  f(g^{-1})$ is a matrix coefficient of the dual representation $(\hat{\pi},V^*)$.
\end{prop}

\begin{proof}
  By the identification $V \simeq V^{**}$, from left to right:
  \[
    \check{f}(g) = f(g^{-1}) = L(\pi(g^{-1})v) = (L \circ \pi(g)^{-1}) v = v^{**}(\hat{\pi}(g))
  \]
  From right to left:
  \[
    f(g) = \check{f}(g^{-1}) = v^{**}(\hat{\pi}(g^{-1})) = (L \circ \pi(g)) v = L(\pi(g)v)
  \]
\end{proof}

Since the vector space associated with matrix coefficients are
finite dimensional, by straightforward linear algebra, we may check the set of
linear functionals $B^* = \{L_i\}_{i = 1}^{n}$ induced from the basis $B =
\{e_i\}_{i = 1}^{n}$ for $V$ is a basis for the dual space $V^*$.  Hence
$\dim(\mathcal{M}_{\pi}) \leq \dim(V)^2$. This allows us to prove the following
theorem.

Let $\lambda(g)f$ and $\rho(g)f$ denote left and right translations by $g$ respectively,
\begin{equation}
  \label{eq:LR-translation}
  \lambda(g)(f) = x \mapsto f(g^{-1}x),
  \qquad
  \rho(g)(f) = x \mapsto f(xg)
\end{equation}

\begin{thm} \label{left-right-finite}
  Let $f$ be a function on $G$. Then the followings are equivalent:
  \begin{enumerate}
    \item $\lambda(g)f$ spans a finite dimensional vector space.
    \item $\rho(g)f$ spans a finite dimensional vector space.
    \item $f$ is a matrix coefficient of a finite dimensional representation.
  \end{enumerate}
\end{thm}

\begin{proof}
  Firstly, suppose (3), let $(\pi,V)$ be a finite dimensional representation of
  $G$ and $f(h) = L(\pi(h)v)$, then $(\lambda(g)(f))(h) = f(g^{-1}x) =
  L(\pi(g^{-1}x)v)$ and $(\rho(g)(f))(h) = f(xg) = L(\pi(xg)v)$, they are also
  matrix coefficients of $V$.  Because $\dim(V) < \infty$,
  $\dim(\mathcal{M}_{\pi}) \leq \dim(V)^2$, the vector spaces spanned by left
  and right translations also have finite dimension, hence $(3) \Rightarrow
  (1),(2)$.

  Suppose $\rho(g)(f)$ spans a vector space $V \subset \Hom(G,\C)$, $\dim(V) <
  \infty$, then $(\rho,V)$ is a representation of $G$: the action is $\rho(g)(v)
  = x \mapsto v(xg)$.  Define a linear functional $L \in V^* \subset
  \Hom(\Hom(G,\C),\C)$ by $L(\phi) = \phi(1)$, where $1$ is the unit of $G$.
  Then $L(\rho(g)f) = (\rho(g)(f))(1) = f(g1) = f(g)$, $f$ is indeed a matrix
  coefficient of $V$, hence $(2) \Rightarrow (3)$.
  
  Finally, suppose $\lambda(g)(f)$ spans a vector space $V \subset \Hom(G,\C)$,
  $\dim(V) < \infty$. Let $\check{f}(h) = f(h^{-1})$ as usual, and define
  $\widetilde{V}$ by
  \begin{align*}
    \widetilde{V} & = \SetForm{\lambda(g)f \circ (-)^{-1} \in \Hom(G,\C)}{g \in G}\\
    & = \SetForm{h \mapsto f(g^{-1}h^{-1})}{g \in G}
    = \SetForm{\rho(g)(\check{f})}{g \in G}
  \end{align*}

  Then $\dim(\widetilde{V}) < \infty$, by the previous argument, $\check{f}$ is
  a matrix coefficient, by \Cref{dual-mat-coeff}, $f$ is a matrix coefficient.
  Hence, $(1) \Rightarrow (3)$.
\end{proof}

\subsection{Schur's Orthogonality}

Before we dive into the discussion of orthogonality of matrix coefficients, we
firstly recall a remarkable theorem in functional analysis that uniquely
determines the forms of continuous linear functionals on a Hilbert space by its
inner product.

\begin{thm}[F. Riesz representation theorem in a Hilbert space] \label{riesz-rep-thm-hilbert}
  Let $(H,\langle \cdot, \cdot \rangle)$ be a Hilbert space over $\R$ or $\C$.  Then, given any
  continuous linear functional $l \in H^*$, there exist one and only one vector
  $y_l \in H$ such that
  \[
    l(x) = \langle x, y_l \rangle \qquad (\forall x \in H)
  \]
\end{thm}

\begin{proof}
  See Theorem 4.6-1 in \cite{ciarlet2013linear}.
\end{proof}

Hence, if $(\pi,H)$ is a finite dimensional representation of compact group $G$,
then all matrix coefficients of $H$ are of the from $g \mapsto L(\pi(g)v) =
\langle \pi(g)v, y_L \rangle$ for some $y_L \in H$.

\begin{lem}\label{schur-con-inter-op}
  Let $G$ be group, $(\pi_1, V_1)$ and $(\pi_2, V_2)$ be (complex)
  representations of $G$. Let $\langle , \rangle$ be any inner product on $V_1$.
  If $v_i,w_i \in V_i$, then the map $T : V_1 \to V_2$ defined by
  \[
    T(w) = \int_G \langle \pi_1(g)w,v_1 \rangle \cdot \pi_2(g^{-1}) v_2 dg
  \]
  is an interwining operator between $V_1$ and $V_2$.
\end{lem}

\begin{proof}
  We want to show, given $h \in G$,
  \[
    \pi_2(h)(T(v)) \stackrel{?}{=} T(\pi_1(h)(v))
  \]
  Let's expand the definition on the right hand side:
  \begin{align*}
    T(\pi_1(h)(v)) & = \int_G \langle \pi_1(g)(\pi_1(h)(v)),v_1 \rangle \cdot \pi_2(g^{-1}) v_2 dg\\
                   & = \int_G \langle \pi_1(gh)(v),v_1 \rangle \cdot \pi_2(g^{-1}) v_2 dg
  \end{align*}
  By change of variable $g \mapsto gh^{-1}$ gives:
  \begin{align*}
    T(\pi_1(h)(v)) & = \int_G \langle \pi_1(g)(v),v_1 \rangle \cdot \pi_2(hg^{-1}) v_2 d(gh^{-1})\\
                   & = \int_G \langle \pi_1(g)(v),v_1 \rangle \cdot \pi_2(h)(\pi_2(g^{-1}) v_2) d(gh^{-1})\\
                   & = \int_G \langle \pi_1(g)(v),v_1 \rangle \cdot \pi_2(h)(\pi_2(g^{-1}) v_2) dg && \text{(right invariance)}\\
                   & = \int_G  \pi_2(h) (\langle \pi_1(g)(v),v_1 \rangle \cdot (\pi_2(g^{-1}) v_2)) dg && \text{(linearity)}\\
                   & = \pi_2(h)(\int_G \langle \pi_1(g)v , v_1  \rangle \cdot \pi_2(g^{-1}) v_2 dg) && \text{(linearity)}\\
                   & = \pi_2(h)(T(v))
  \end{align*}
\end{proof}

\begin{thm}[Schur orthogonality, between representations] \label{schur-ortho-1}
  Let $G$ be a compact group, $(\pi_1, V_1)$, $(\pi_2, V_2)$ be two irreducible
  representations of $G$.  Then either $\mathcal{M}_{\pi_1} \perp
  \mathcal{M}_{\pi_2}$ in $L^2(G)$ or the representations are isomorphic.
\end{thm}

\begin{proof}
  We will prove the result by negating one of the conclusion and use it to prove
  the other.  Suppose $\mathcal{M}_{\pi_1} \not\perp \mathcal{M}_{\pi_2}$, then
  there exist matrix coefficients $f_1 \in \mathcal{M}_{\pi_1}$ and $f_2 \in
  \mathcal{M}_{\pi_2}$ such that $\langle f_1 , f_2 \rangle \neq 0$.  By Riesz
  representation theorem, we may characterize $f_1$ and $f_2$ by:
  \[
    f_1(g) = \langle \pi_1(g)w_1 , v_1 \rangle
    \quad
    f_2(g) = \langle \pi_2(g)w_2 , v_2 \rangle
  \]
  for some $w_i, v_i \in V_i$.  To avoid notational confusion, let
  $H(\cdot,\cdot)$ denote the inner product of the Hilbert space, and let
  $\langle \cdot, \cdot \rangle$ denote the inner product of both $V_1$ and
  $V_2$. Then by our assumption,
  \[
    H(f_1, f_2) = \int_G \langle \pi_1(g)w_1, v_1 \rangle \cdot
    \conj{\langle \pi_2(g)w_2, v_2 \rangle} dg \neq 0
  \]
  By complex conjugate, invariance and linearity,
  \begin{align*}
    & \int_G \langle \pi_1(g)w_1, v_1 \rangle \cdot \conj{\langle \pi_2(g)w_2, v_2 \rangle} dg\\
    & = \int_G \langle \pi_1(g)w_1, v_1 \rangle \cdot \langle v_2, \pi_2(g)w_2 \rangle dg\\
    & = \int_G \langle \pi_1(g)w_1, v_1 \rangle \cdot \langle \pi_2(g^{-1}) v_2, \pi_2(g^{-1}) \pi_2(g)w_2 \rangle dg\\
    & = \int_G \langle \pi_1(g)w_1, v_1 \rangle \cdot \langle \pi_2(g^{-1}) v_2, w_2 \rangle dg\\
    & = \int_G \langle \langle \pi_1(g)w_1, v_1 \rangle \cdot \pi_2(g^{-1}) v_2, w_2 \rangle dg\\
    & = \left\langle \int_G \langle \pi_1(g)w_1, v_1 \rangle \cdot \pi_2(g^{-1}) v_2 dg, w_2 \right\rangle \neq 0
  \end{align*}
  Let $T : V_1 \to V_2$ be define as in \Cref{schur-con-inter-op} and plugging
  the definition, we have,
  \[
    \langle T(w_1),w_2 \rangle \neq 0
  \]
  Hence $T : V_1 \to V_2$ is not zero, by Schur's Lemma (in compact case), $T$
  is an isomorphism.
\end{proof}

\Cref{schur-ortho-1} shows that if we have two non-isomorphic irreducible
representations of a compact group $G$, then any pair of matrix coefficients of
each representation is orthogonal.  We now consider the orthogonality of matrix
coefficients of the same irreducible representation, we will give an explicit
formula for computing the inner product of matrix coefficients of the same
irreducible representation.

\begin{thm}[Schur's Orthogonality, in one representation] \label{schur-ortho-2}
  Let $G$ be a compact group, $(\pi,V)$ be an irreducible representation of $G$.
  $V$ is equipped with inner product $\langle \cdot,\cdot \rangle$.
  Then fixing $v_1,v_2,w_1,w_2 \in V$, there exist a constant $d > 0$ such that
  \[
    \int_G \langle \pi(g)w_1, v_1 \rangle \cdot \conj{\langle \pi(g)w_2, v_2 \rangle} dg
    = d^{-1} \langle w_1, w_2 \rangle \cdot \langle v_2, v_1 \rangle
  \]
\end{thm}

\begin{proof}
  Firstly, let $v_1, v_2$ be fixed.  Let $T : V \to V$ be defined similar to
  \Cref{schur-con-inter-op} by:
  \[
    T(w) = \int_G \langle \pi(g)w, v_1 \rangle \pi(g^{-1})v_2 dg
  \]
  Then, by Schur's Lemma, $T = \lambda I$ for some constant $\lambda$ (depending
  on $v_1$ and $v_2$), and
  \begin{align*}
      & \int_G \langle \pi(g)w_1, v_1 \rangle \cdot \conj{\langle \pi(g)w_2, v_2 \rangle} dg\\
    = & \int_G \langle \pi(g)w_1, v_1 \rangle \cdot \langle v_2, \pi(g)w_2\rangle dg\\
    = & \int_G \langle \pi(g)w_1, v_1 \rangle \cdot \langle \pi(g^{-1})v_2, w_2\rangle dg\\
    = & \left\langle \int_G \langle \pi(g)w_1, v_1 \rangle \cdot \pi(g^{-1})v_2 dg , w_2 \right\rangle\\
    = & \langle T(w_1, w_2) \rangle = \lambda \langle w_1,w_2 \rangle
  \end{align*}
  Now, let $w_1, w_2$ be fixed. Let $\widetilde{T} : V \to V$ be defined by
  \[
    \widetilde{T}(v) = \int_G \langle \pi(g)v, w_2 \rangle \cdot \pi(g^{-1})w_1 dg
  \]
  Then, $\widetilde{T} = \widetilde{\lambda} I$, and
  \begin{align*}
      & \int_G \langle \pi(g)w_1, v_1 \rangle \cdot \conj{\langle \pi(g)w_2, v_2 \rangle} dg\\
    = & \int_G \langle \pi(g)w_1, v_1 \rangle \cdot \langle v_2, \pi(g)w_2 \rangle dg\\
    = & \int_G \langle \pi(g)w_1, v_1 \rangle \cdot \langle \pi(g^{-1})v_2, w_2 \rangle dg
  \end{align*}
  Substitute $g$ by $g^{-1}$, since Haar measures on compact groups are unique
  upto a constant multiple, let $c$ be the constant, then $d(g^{-1}) = cdg$,
  hence
  \begin{align*}
    \ldots = & c \int_G \langle \pi(g^{-1})w_1, v_1 \rangle \cdot \langle \pi(g)v_2, w_2 \rangle dg\\
    = & c \int_G \langle \langle \pi(g)v_2 w_2 \rangle \pi(g^{-1})w_1 , v_1 \rangle dg\\
    = & c \langle \int_G \langle \pi(g)v_2 w_2 \rangle \pi(g^{-1})w_1 dg , v_1 \rangle\\
    = & c \langle \widetilde{T}(v_2), v_1 \rangle = c \widetilde{\lambda} \langle v_2, v_1 \rangle
  \end{align*}
  Combining the two results above, there exist a constant $d$ such that
  \[
    \int_G \langle \pi(g)w_1, v_1 \rangle \cdot \conj{\langle \pi(g)w_2, v_2 \rangle} dg
    = d^{-1} \langle w_1, w_2 \rangle \cdot \langle v_2, v_1 \rangle
  \]
  If we let $w_1 = w_2$, $v_1 = v_2$, then the left hand side is positive the
  definition of inner product on the Hilbert space, then $d$ must be positive.
\end{proof}



% \subsection{Representation of Compact Groups}

% \begin{lem}[Complete reducibility]
%   If $G$ is compact, then each finite dimensional representation is the direct
%   sum of irreducible representations.
% \end{lem}

% \begin{proof}
  
% \end{proof}

% \begin{thm}[Schur orthogonality]
%   Suppose $(\pi_1, V_1)$ and $(\pi_2, V_2)$ are irreducible representations of a
%   compact group $G$. Either every matrix coefficient of $\pi_1$ is orthogonal in
%   $L^2(G)$ to every matrix coefficient coefficient of $\pi_2$, or the
%   representations are isomorphic.
% \end{thm}

% \begin{proof}
% \end{proof}

% \begin{defn}[Character, for compact group]

% \end{defn}


\subsection{Peter-Weyl's Theorem}

The goal of the this main section is to prove an arbitrary representation of a
compact group can be reduced into a direct sum of \emph{finite dimensional}
irreducible representations.  In particular, if we have the reducibility, by
\Cref{schur-ortho-1} and \Cref{schur-ortho-2}, we can show the components are
pairwise orthogonal.

The main technique used in the proofs follows from the idea of using matrix
coefficients to construct suitable finite dimensional subrepresentations of the
given representation.  Thus, what lies in the heart of Peter-Weyl's Theorem is
the observation that there exist an ``adequate'' supply of matrix coefficients
on the given group $G$.  Hence, we sometimes refer to this result
(\Cref{peter-weyl-1}) as the Peter-Weyl's Theorem.


Throughout the discussion of Peter-Weyl's Theorem, we will be mainly interested
in normed vector space of continuous functions on $G$, or space of $L^p$
functions on $G$, where the $p$-norm is defined as, for $1 \leq p < \infty$
\[
  \norm{f}_p \triangleq \left\{ \int_G \abs{f(g)}^{p}   \right\}^{1/p}
\]
when $p = \infty$, we define the sup-norm by
\[
  \norm{f}_{\infty} \triangleq \esssup_G f = \inf \SetForm{a \in \R}{\mu\SetForm{g \in G}{f(x) > a} = 0}
  = \sup_{G} \abs{f(x)} \quad \text{if } f \in C(G)
\]
We define $L^p(G)$ to be the set containing all functions $f$ on $G$ such that
$\norm{f}_p < \infty$.

Let $C(G)$ denote the collection of continuous functions on compact group $G$,
then $C(G)$ forms a ring with multiplication being \emph{convolution}:
\[
  (f_1 \star f_2)(g) = \int_G f_1(gh^{-1}) f_2(h) dh = \int_G f_1(h) f_2(h^{-1}g) dh
\]

Given $\phi \in C(G)$, let $T_{\phi} : C(G) \to C(G)$ be the linear operator by
left convolution: $T_{\phi} : f \mapsto \phi \star f$.

\begin{prop} \label{1-2-infty}
  Let $f \in C(G)$ be given, then
  \[
    \norm{f}_1 \leq \norm{f}_2 \leq \norm{f}_{\infty}
  \]
\end{prop}

\begin{proof}
  By Cauchy-Schwarz inequality, let $\langle \cdot , \cdot \rangle$ be the inner
  product defined as \Cref{Hilbert-inner-product}, let $1$ denote the constant
  function $1$,
  \[
    \norm{f}_1 = \langle \abs{f} , 1 \rangle
    \leq \left(\langle \abs{f}, \abs{f} \rangle \right)^{1/2} \cdot \left( \langle 1, 1 \rangle \right)^{1/2}
    = \left( \int_G \abs{f(g)}^2 dg \right)^{1/2} = \norm{f}_2
  \]
  The second inequality is trivial,
  \[
    \norm{f}_2
    = \left( \int_G \abs{f(g)}^2 dg \right)^{1/2}
    \leq \left( \int_G \norm{f}_{\infty}^{2} dg \right)^{1/2}
    = \norm{f}_{\infty}
  \]
\end{proof}

\begin{prop} \label{Tphi-bounded}
  Let $G$ be a compact group, $\phi \in C(G)$.  Then $T_{\phi}$ is a bounded
  operator on $L^1(G)$. Further, if $f \in L^1(G)$, then $T_{\phi}f \in
  L^{\infty}(G)$ and
  \[
    \norm{T_{\phi}f}_{\infty} \leq \norm{\phi}_{\infty} \norm{f}_1
  \]
\end{prop}

\begin{proof}
  Let $f \in L^1(G)$ be given. We estimate the sup-norm of $T_{\phi}f$ by
  \[
    \norm{T_{\phi} f}_{\infty}
    = \sup_{g \in G} \abs{\int_G \phi(g h^{-1}) f(h) dh}
    \leq \norm{\phi}_{\infty} \int_G \abs{f(h)} dh
  \]
  Put $C \triangleq \norm{\phi}_{\infty}$, $C < \infty$ since $\phi$ is a
  continuous function on a compact set hence obtain a finite extrema.  Since $f
  \in L^1(G)$, $\norm{f}_1 < \infty$, thus $T_{\phi}f \in L^{\infty}(G)$.
  
  Furthermore by \Cref{1-2-infty}, $\norm{T_{\phi} f}_1 \leq \norm{T_{\phi} f}_2
  \leq \norm{T_{\phi} f}_{\infty}$,  We have, $\norm{T_{\phi}f}_1 \leq C
  \norm{f}_1$, thus by definition the operator $T_{\phi}$ is bounded in
  $L^1(G)$. In fact, $T_{\phi}$ is bounded in all three norms: $1,2,\infty$.
\end{proof}

\begin{prop}\label{self-adjoint-compact}
  Let $G$ be a compact group, $\phi \in C(G)$, then
  \begin{enumerate}
    \item $T_{\phi}$ is a bounded operator on $L^2(G)$ and $\abs{T_{\phi}} \leq \supnorm{\phi}$.
    \item $T_{\phi}$ is compact.
    \item If $\phi(g^{-1}) = \conj{\phi(g)}$, it is self-adjoint.
  \end{enumerate}
\end{prop}

\begin{proof}
  \begin{enumerate}
    \item Let $f \in L^2(G)$ be given, then $f \in L^1(G)$ by \Cref{1-2-infty}
      and by the argument in \Cref{Tphi-bounded},
      \[
        \norm{T_{\phi} f}_{2} \leq
        \supnorm{T_{\phi} f} \leq
        \supnorm{\phi} \norm{f}_1 \leq \supnorm{\phi} \norm{f}_2
      \]
      Hence $T_{\phi}$ is bounded on $L^2(G)$ and $\abs{T_{\phi}} \leq \supnorm{\phi}$.

    \item 
      Let $B$ be a set of bounded functions in $L^2(G)$, without loss of
      generality, we consider the unit ball in $L^2(G)$. Since $L^2(G) \subset
      L^1(G)$, it suffices to consider the unit ball in $L^1(G)$ , that is:
      \[
        B \triangleq \SetForm{f \in L^1(G)}{\norm{f}_1 \leq 1}
      \]
      We want to show the image set $T_{\phi}(B)$ is sequentially compact, that
      is, every infinite sequence in $T_{\phi}(B)$ has a convergent subsequence.
      We will establish the result by using Ascoli and Arzela
      \Cref{ascoli-arzela}.  First off, by \Cref{Tphi-bounded}, we know
      $T_{\phi}(B)$ is bounded, hence it suffice to show $T_{\phi}(B)$ is
      equicontinuous.

      Let $\epsilon > 0$ be given, since $\phi \in C(G)$ and $G$ is compact,
      $\phi$ is uniformly continuous, there exist a neighborhood $N$ of the
      identity $1_G \in G$ such that
      \[
        \abs{\phi(kg) - \phi(g)} < \epsilon \qquad (\forall k \in N)
      \]
      Then given $f \in B$, $k \in N$, $g \in G$,
      \begin{align*}
        \abs{T_{\phi}f(kg) - T_{\phi}f(g)}
        & = \abs{\int_G (\phi(kgh^{-1}) - \phi(gh^{-1})) f(h) dh}\\
        & \leq \int_G \abs{\phi(kgh^{-1}) - \phi(gh^{-1})} \cdot \abs{f(h)} dh\\
        & \leq \epsilon \norm{f}_1 \leq \epsilon
      \end{align*}
      By definition, $T_{\phi}(B)$ is equicontinuous, hence apply
      \Cref{ascoli-arzela}, $T_{\phi}(B)$ is sequentially compact, $T_{\phi}$ is
      compact.

    \item Suppose we have $\phi(g^{-1}) = \conj{\phi(g)}$, then
      \begin{align*}
        \langle T_{\phi}f_1, f_2 \rangle
        & = \int_G T_{\phi}f_1(g) \cdot \conj{f_2(g)} dg\\
        & = \int_G (\int_G \phi(gh^{-1}) f_1(h) dh) \cdot \conj{f_2(g)} dg\\
        & = \int_G (\int_G \conj{\phi(hg^{-1})} f_1(h) dh) \cdot \conj{f_2(g)} dg\\
        & = \int_G f_1(h) \cdot \conj{T_{\phi}f_2(h)} dh\\
        & = \langle f_1, T_{\phi}f_2 \rangle
      \end{align*}
      Hence $T_{\phi}$ is self-adjoint.
  \end{enumerate}
\end{proof}

\begin{prop} \label{Tphi-invariant}
  Let $G$ be a compact group, $\phi \in C(G)$, $\lambda \in \C$, then the
  $\lambda$-eigenspace
  \[
    V(\lambda) = \SetForm{f \in L^2(G)}{T_{\phi}f = \lambda f}
  \]
  is invariant under $\rho(g)$ for all $g \in G$, where $\rho(g)$ is defined as
  usual as in \Cref{eq:LR-translation}.
\end{prop}

\begin{proof}
  Let $g \in G$ be given,
  \begin{align*}
    (T_{\phi}(\rho(g)f))(x)
    & = \int_G \phi(xh^{-1}) f(hg) dh\\
    & = \int_G \phi(xgh^{-1}) f(h) dh && \text{substitute } h \to hg^{-1}\\
    & = (\rho(g)(T_{\phi}f))(x)
  \end{align*}
\end{proof}

\begin{thm}[Peter-Weyl, Part 1] \label{peter-weyl-1}
  The matrix coefficients of $G$ are dense in $L^2(G)$.
\end{thm}

\begin{proof}
  Let $f \in L^2(G)$ be given, we need to show for any $\epsilon > 0$, there
  exist a matrix coefficient $f'$ (thus associated with a finite dimensional
  representation), such that $\norm{f' - f} < \epsilon$.

  Let $\epsilon > 0$ be given, since $G$ is compact, $f$ is uniformly continuous
  on $G$, hence left translation of $f$ by $g$: $\lambda(g)f = h \mapsto
  f(g^{-1}h)$ is uniformly continuous on $G$. Then there exist a neighborhood $U
  \ni 1_G$ such that $\norm{\lambda(g)f - \lambda(1_G)f}_{\infty} =
  \norm{\lambda(g)f - f}_{\infty} < \epsilon/2$ for all $g \in U$.

  Define $\phi : U \to \C$ be a continuous map satisfying $\int_G \phi(g) dg = \int_U
  \phi(g) dg = 1$, and $\phi(g) = \phi(g^{-1})$.  Then the operator $T_{\phi}$
  is self-adjoint and compact by \Cref{self-adjoint-compact}. We collect the
  following facts:
  \begin{enumerate}
    \item $\norm{T_{\phi}f - f} < \epsilon/2$:
      \begin{align*}
        \abs{T_{\phi}f(h) - f(h)}
        & = \abs{\int_G \phi(g) f(g^{-1}h) - \phi(g)f(h) dg}\\
        & = \abs{\int_U \phi(g) f(g^{-1}h) - \phi(g)f(h) dg}\\
        & \leq \int_U \phi(g) \cdot \abs{f(g^{-1}h) - f(h)} dg\\
        & \leq \int_U \phi(g) \cdot \norm{\lambda(g)f - f}_{\infty} dg
          \leq \frac{\epsilon}{2}
      \end{align*}

    \item $T_{\phi}$ is a compact operator on $L^2(G)$ by \Cref{self-adjoint-compact}.

    \item 
      Let $\lambda$ denote an eigenvalue of $T_{\phi}$, by spectral theorem
      \Cref{spectral}, the eigenspaces $V(\lambda) \subset L^2(G)$ are all
      finite dimensional except perhaps $V(0)$.  Furthermore, the spectral
      theorem tells us spaces $V(\lambda)$ are mutually orthogonal and they span
      $L^2(G)$.

    \item 
      By \Cref{Tphi-invariant}, maps in $V(\lambda)$ are all $T_{\phi}$
      invariant.
  \end{enumerate}

  Let $f_\lambda$ be the projection of $f$ on the basis $V(\lambda)$:
  \[
    f_{\lambda} = \sum_{w \in V(\lambda)} \langle f, w \rangle w
  \]
  By orthogonality between $V(\lambda)$s, we have
  \[
    \sum_{\lambda} \norm{f_{\lambda}}_2^2 = \norm{f}_2^2 < \infty
  \]
  For some positive real number $q$, define $f', f''$ by:
  \[
    f'' \triangleq \sum_{\abs{\lambda} > q} f_{\lambda},
    \qquad
    f' \triangleq T_{\phi}f''
  \]
  Then both $f'$ and $f''$ are contained in a finite dimensional vector space:
  $\bigoplus_{\abs{\lambda} > q} V(\lambda)$.  By (4), and apply
  \Cref{left-right-finite}, it follows $f$ is a matrix coefficient of a finite
  dimensional representation.

  Lastly, we will show $f'$ indeed approximates $f$ within $\epsilon$: choose
  $q$ so that
  \[
    \norm{\sum_{0 < \abs{\lambda} < q} f_{\lambda}}_1
    \leq \norm{\sum_{0 < \abs{\lambda} < q} f_{\lambda}}_2
    = \sqrt{\sum_{0 < \abs{\lambda} < q} \norm{f_{\lambda}}_2^2}
    < \frac{\epsilon}{2\supnorm{\phi}}
  \]
  Then,
  \begin{align*}
    \supnorm{T_{\phi}(f - f'')}
    & = \supnorm{T_{\phi} \left( f_0 + \sum_{0 < \abs{\lambda} < q} f_{\lambda} \right)}
    = \supnorm{T_{\phi} \left( \sum_{0 < \abs{\lambda} < q} f_{\lambda} \right)}\\
    & \leq \supnorm{\phi} \cdot \norm{\sum_{0 < \abs{\lambda} < q} f_{\lambda}}_1
    < \frac{\epsilon}{2}
  \end{align*}
  Therefore, use $T_{\phi}$ invariance,
  \[
    \supnorm{f - f'}
    = \supnorm{f + T_{\phi}(f - f - f'')}
    = \supnorm{f - T_{\phi}f} + \supnorm{T_{\phi}f - T_{\phi}f''}
    < \frac{\epsilon}{2} + \frac{\epsilon}{2} = \epsilon
  \]
\end{proof}

\begin{cor}
  The matrix coefficients of compact group $G$ are dense in $L^2(G)$.
\end{cor}

\begin{proof}
  Since $C(G)$ is dense in $L^2(G)$, it follows from \Cref{peter-weyl-1}.
\end{proof}

% \begin{thm}
% \end{thm}

% \begin{proof}
% \end{proof}

\begin{thm}[Peter-Weyl, Part 2] \label{peter-weyl-2}
  Let $H$ be a Hilbert space and $G$ be a compact group. Let $\pi : G \to
  End(H)$ be a unitary representation. Then $H$ is a direct sum of
  finite dimensional irreducible representations.
  % The space of square integrable functions on $G$ is the Hilbert space direct
  % sum over finite dimensional irreducible representations $V$,
  % \[
  %   \ell^2(G) \cong \hat{\bigoplus} End(V)
  % \]
\end{thm}

\begin{proof}
  We will prove firstly, if $H$ is nonzero then it has an irreducible finite
  dimensional invariant subspace. Then, we will extract a finite dimensional
  representation of $G$ on $H$.

  Let $v \in H$ be a nonzero vector, since the group action $\pi$ is continuous
  and $G$ is compact, it's uniformly continuous in $G$. Hence there exist a
  neighborhood $N$ of $1_G \in G$ such that
  \[
    \norm{\pi(g)v - \pi(1_G)v} = \norm{\pi(g)v - v} \leq \frac{\norm{v}}{2} \qquad (\forall g \in N)
  \]
  Define $\phi : N \to \C$ be a continuous map satisfying $\int_G \phi(g) dg =
  1$. Then
  \[
    \left\langle \int_G \phi(g) \cdot \pi(g)v dg , v \right\rangle
    = \langle v, v \rangle - \left\langle \int_N \phi(g) \cdot (v - \pi(g)v) dg, v \right\rangle
  \]
  and by Cauchy Schwarz,
  \begin{align*}
    & \norm{ \left\langle \int_N \phi(g) \cdot (v - \pi(g)v) dg, v  \right\rangle }\\
    & \leq \norm{\int_N \phi(g) \cdot (v - \pi(g)v) dg} \cdot \norm{v}\\
    & \leq \int_N \norm{\phi(g)} \norm{v - \pi(g)v} dg \cdot \norm{v}\\
    & \leq \int_N \norm{\phi(g)} \cdot \frac{\norm{v}}{2} dg \cdot \norm{v}
      = \frac{\norm{v}^2}{2}
  \end{align*}
  Hence, the integral $\int_G \phi(g) \cdot \pi(g)v dg \neq 0$.

  By \Cref{peter-weyl-1}, since $\phi \in C(G)$, for arbitrary $\epsilon > 0$,
  there exist a matrix coefficient $f$ associated to a finite dimensional
  representation $(\rho, W)$ such that $\supnorm{f - \phi} < \epsilon$. Thus, in
  this case,
  \[
    \norm{\int_G (f - \phi)(g) \cdot \pi(g)v dg} \leq \epsilon \norm{v}
  \]
  Take small enough $\epsilon$, then $\norm{\int_G f(g) \cdot \pi(g)v dg} =
  \norm{\int_G \phi(g) \cdot \pi(g)v dg} + \epsilon \norm{v} \neq 0$.

  By \Cref{dual-mat-coeff}, $\check{f}$ is a matrix coefficient of the dual
  representation $(\hat{\rho}, W^*)$ (which has the same dimension). By
  definition of matrix coefficient, $\check{f}(g) = L(\hat{\rho}(g)w)$ for some $L \in
  (W^* \to \C) \simeq W$, $w \in W^*$. Define $T : W^* \to H$ by
  \[
    T(x) \triangleq \int_G L(\hat{\rho}(g^{-1})x) \cdot \pi(g)v dg
  \]
  using the same argument used to prove \Cref{schur-con-inter-op}, we know $T
  \in \Hom^G(W^*, H)$ and it's nonzero since when evaluated at $w$:
  \[
    T(w) = \int_G L(\hat{\rho}(g^{-1})w) \cdot \pi(g)v dg
    = \int_G \check{f}(g^{-1}) \cdot \pi(g)v dg
    = \int_G f(g) \cdot \pi(g)v dg \neq 0
  \]
  Since $\dim(W^*) < \infty$, the image $T(W^*)$ is a nonzero finite dimensional
  invariant subspace of $H$!

  Now, let $\Sigma$ be the set of all sets of finite dimensional irreducible
  invariant subspaces, order its elements by inclusion.  By Zorn's Lemma, the
  partial order $\Sigma$ has a maximal element $\widetilde{S}$.  If
  $\widetilde{S}$ spans $H$, we are done with $\widetilde{S}$ being a complete
  decomposition of $H$.  Otherwise, the complement of the span of $S$ is a
  nonzero, by previous arguments, it contains an invariant irreducible subspace
  $U$, then $\widetilde{S}$ is not maximal since $\widetilde{S} \cup \{U\}$ is
  larger than $\widetilde{S}$, we have a contradiction!
\end{proof}

We have shown that a Hilbert space representation of a compact group may be
completely reduced into a direct sum of \emph{finite dimensional} irreducible
representations.  If we instantiate the result with $G = \T$, $H = L^2(\R)$ and
an action $\widetilde{\pi} : \T \times L^2(\R) \to L^2(\R)$
\[
  \widetilde{\pi}(e^{i\theta})f \triangleq e^{i\theta} \cdot \hat{f}
\]
where $\hat{f}$ is the complete decomposition of $f$ into orthogonal components
given by Peter-Weyl's Theorem (which is exactly the Fourier transform of $f$).
If we chain $\widetilde{\pi}$ together with the quotient map $\R \to \R/\Z
\simeq \T$ that sends $t \in \R$ to $e^{it} \in \T$, then we recover the map
$\pi : \R \times L^2(\R) \to L^2(\R)$ from the introduction.
