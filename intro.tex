\section{Introduction}

Let $f : [-\pi,\pi] \to \R$ be a $2\pi$ periodic function, we may associate to it with a
sequence \emph{Fourier coefficients} $\hat{f} : \Z \to \R$, defined by:
\[
  \hat{f}(n) \triangleq \frac{1}{2\pi} \int_{-\infty}^{\infty} f(\theta) e^{-in\theta} d\theta
\]
and approximate the original function by the \emph{Fourier series} of $f$:
\[
  f(\theta) \sim \sum_{-\infty}^{\infty} \hat{f}(n) e^{in\theta}
\]
If $f \in L^2(-\pi, \pi)$, by Theorem 8.43 in \cite{folland}, we can establish a
correspondence between the series and $f$ by showing the series converges to $f$
in norm (of the $L^2$ space), that is,
\[
  \lim_{N \to \infty} \int_{-\pi}^{\pi} \abs{f(\theta) - \sum_{-N}^{N} \hat{f}(n)
    e^{in\theta}} = 0
\]
Moreover, since the integral of $e^{in\theta}$ from $\theta = -\pi$ to $\pi$ is zero
except for $n = 0$, $\langle e^{in\theta}, e^{im\theta} \rangle = 0$ for all $m
\neq n$. Hence $\{e^{in\theta}\}_{-\infty}^{\infty}$ forms an \emph{orthogonal
  basis} for $L^2(-\pi,\pi)$.

The first generalization one may make is to allow $f$ to be a (possibly)
non periodic function from $\R$ to $\R$, and consider the ``continuous''
analogue of Fourier coefficients.  This generalization defines us the
\emph{Fourier transform} of $f$, $\mathcal{F}f = \hat{f} : \R \to \R$:
\[
  \hat{f}(\xi) \triangleq \frac{1}{2\pi} \int_{-\infty}^{\infty} f(\theta)
  e^{-i\xi\theta} d\theta
\]
and the corresponding approximation for the original $f$:
\begin{equation}
  \label{eq:fourier-inversion}
  f(\theta) \sim \int_{-\infty}^{\infty} \hat{f}(\xi) e^{i\xi\theta} d\xi
\end{equation}
If we restrict $f \in L^2(\R)$, then by Theorem 1.9 in Chapter 5 of
\cite{stein2011fourier}, \cref{eq:fourier-inversion} turns out to be an
equality, hence the name: \emph{Fourier inversion} formula. By the same argument
as before, the set $\{e^{i\xi\theta}\}$ is an orthogonal basis of $L^2(\R)$.

The second and the third possible generalization follows from the observation
that the Fourier transform $\mathcal{F}(-)$ is a endomorphism on the group
$L^2(\R)$ under convolution.  If we define an operator $T_t : L^2(\R) \to
L^2(\R)$, $(T_tf)(\theta) \triangleq f(\theta + t)$ for every $t \in \R$, and
define the group action $\pi : \R \times L^2(\R) \to L^2(\R)$ by $\pi(t)f =
\mathcal{F}(T_tf)$. A simple calculation by substitution shows, for a given $t
\in \R$,
\[
  \widehat{T_t f}(\xi)
  = \frac{1}{2\pi} \int_{-\infty}^{\infty} f(\theta + t) e^{-i\xi\theta} d\theta
  = \frac{1}{2\pi} \int_{-\infty}^{\infty} f(\theta) e^{-i\xi(\theta - t)} d\theta
  = \hat{f}(\xi) \cdot e^{it\xi}
\]
Hence $(L^2(\R), \pi)$ is a representation of $\R$ in the inner product space
$L^2(\R)$ (these terms will be defined more precisely in Section 3).

If we replace $\R$ by a finite group $G$ and the inner product space $L^2(\R)$
by a generic finite-dimensional vector space $V$, we obtain what's called
\emph{discrete Fourier transform}.  Given this generalization, we are also
interested in questions arose classic Fourier theory, that is, (1) whether it is
possible to find a decomposition of elements in $V$ into linear combinations of
a basis and (2) whether an orthogonal basis exist.

The main portion of the paper will be devoted to investigate the same question
for the case where $G$ is an non-abelian compact group and the vector space $V$
is a Hilbert space.

% And the third
% generalization 


% the
% function space $L^2(\R)$ is a group representation of the group additive group
% $\R$ with the Fourier transform $\mathcal{F}(-)$ being the group action.



% The purpose of this paper is to discuss Peter-Weyl's Theorem, a classical result
% in representation theory that generalizes the theory Fourier Transformation.  The
% theorem and machineries all follows from the observation that $L^2(\T)$, the
% space of square integrable functions on the circle group, serves as a group
% representation of $\T$.

% Peter-Weyl's Theorem, when instantiated with group $\T$, asserts the
% representation $L^2(\T)$ may be decomposed into a direct sum of
% finite-dimensional irreducible representations. Moreover, the matrix
% coefficients of the irreducible unitary representations form an orthonormal
% basis of $L^2(\T)$, namely $\{e^{2\pi i n x}\}_{n \in \mathbb{Z}}$.

In Section 2, we will develop background machineries from measure theory and
functional analysis that allow us to pull off the main result in Section 4.  In
Section 3, we will investigate the second generalization of abstract Fourier
analysis, namely discrete Fourier analysis, and in Section 4 we will establish
the result of abstract Fourier analysis on compact groups.

% We will start by introducing background definitions and developing results in
% representation theory that allow us to discuss abstract ``Fourier Analysis'' as
% group actions on arbitrary vector spaces.  In Section 2, we discuss such
% representations on finite groups and we will extend those theories to
% representations of c
% ompact groups in Section 3.  Following that is a proof of
% Peter-Weyl's Theorem in Section 4.  Finally, we will discuss a categorification of
% the abstraction process we've gone through and show the equivalence between
% category of representations of $G$ and category of vector spaces parametrized by
% the dual group $\hat{G}$ of characters.

% % At the end of the paper, we will also discuss a categorification of the result
% % that establishes an equivalence between the category of representations of $G$
% % and the category of 

