% \documentclass[a4paper,12pt,twoside]{amsart}
\documentclass{amsart}
% \documentclass{article}


% \topmargin=+3pt

% \headsep=18pt

% \oddsidemargin=-6pt
% \evensidemargin=-6pt

% \renewcommand{\baselinestretch}{1.2}

% \textwidth=467pt \textheight=655pt

\usepackage{amsmath,amssymb,amsthm,mathtools,paralist,bm,mathrsfs,bm}
\usepackage{
  nameref, % \nameref
  hyperref, % \autoref
  cleveref, % \cref
}

\usepackage[lite]{amsrefs}

\usepackage{notation/modes}

\usepackage[french,english]{babel}
\usepackage[autostyle]{csquotes}

% \usepackage{unicode-math}
% \setmainfont{Minion Pro}
% \setmathfont{Latin Modern Math}
% \setmonofont{Iosevka Slab}

\usepackage{tikz}
\usetikzlibrary{cd}
% \usepackage{pgfplots}

\newtheorem{thm}{Theorem}[section]
\newtheorem{prop}[thm]{Proposition}
\newtheorem{cor}[thm]{Corollary}
\newtheorem{lem}[thm]{Lemma}
\newtheorem{prob}[thm]{Problem}
\newtheorem{innercustomthm}{Problem}
\newenvironment{xprob}[1]
  {\renewcommand\theinnercustomthm{#1}\innercustomthm}
  {\endinnercustomthm}

\theoremstyle{definition}
\newtheorem{defn}[thm]{Definition}

\theoremstyle{remark}
\newtheorem{rem}[thm]{Remark}
\newtheorem{exa}[thm]{Example}

\numberwithin{equation}{section}

\newcommand{\comment}[1]{}
\newcommand{\uncomment}[1]{#1}

\newenvironment{solu}{\begin{proof}[Solution]}{\end{proof}}

% notations
\DeclarePairedDelimiter\ceil{\lceil}{\rceil}
\DeclarePairedDelimiter\floor{\lfloor}{\rfloor}

\newcommand{\uto}{\stackrel{u}{\to}}
\newcommand{\norm}[1]{\left\lVert#1\right\rVert}
\newcommand{\supnorm}[1]{\left\lVert#1\right\rVert_{\infty}}
\newcommand{\abs}[1]{\left|#1\right|}
\newcommand*{\mybox}[1]{\framebox{#1}}
\renewcommand\enquote[1]{\foreignquote{french}{#1}}

\newcommand\SetForm[2]{\left\{ #1 : #2 \right\}}
\newcommand*\conj[1]{\overline{#1}}
\newcommand\closure[1]{\overline{#1}}
\newcommand*\mean[1]{\bar{#1}}
\newcommand{\size}[1]{\#{#1}}

\newcommand{\vsimeq}{\mathbin{\rotatebox[origin=c]{90}{$\simeq$}}}
\newcommand{\vsim}{\mathbin{\rotatebox[origin=c]{90}{$\sim$}}}

\DeclareMathOperator{\Log}{Log}
\DeclareMathOperator{\Arg}{Arg}
\DeclareMathOperator{\C}{\mathbb{C}}
\DeclareMathOperator{\R}{\mathbb{R}}
\DeclareMathOperator{\N}{\mathbb{N}}
\DeclareMathOperator{\Z}{\mathbb{Z}}
\DeclareMathOperator{\T}{\mathbb{T}}
\DeclareMathOperator{\Mero}{Mer}

\DeclareMathOperator{\Hom}{Hom}
\DeclareMathOperator{\End}{End}
\DeclareMathOperator{\Tr}{Tr}
\DeclareMathOperator{\Ima}{Im}

\DeclareMathOperator{\dist}{dist}

\DeclareMathOperator{\esssup}{esssup}

\newcommand{\Res}[2]{\text{Res}\left[ #1 , #2 \right]}

\newcommand*\Laplace{\mathop{}\!\mathbin\bigtriangleup}
\newcommand*\DAlambert{\mathop{}\!\mathbin\Box}

\DeclareMathAlphabet{\mathcal}{OMS}{cmsy}{m}{n}

% \oddsidemargin-1mm
% \evensidemargin-0mm
% \topmargin-15mm

% \textwidth6.5in
% \headsep25pt
% \textheight8.75in
% \footskip27pt

\pagestyle{headings}

% \renewcommand{\theenumi}{(\alph{enumi}}

\DeclareMathOperator{\bv}{\bm{v}}
\DeclareMathOperator{\bu}{\bm{u}}
\DeclareMathOperator{\bw}{\bm{w}}

\begin{document}

% \title{Peter-Weyl's Theorem (draft)}
\author{Thomas S. Lou}
\title{Abstract Fourier Analysis}
\address{University of Washington}
\email{lous\char`\@cs.washington.edu}
\date{\today}
\maketitle

\begin{abstract}
In this paper, we will discuss abstract Fourier analysis on arbitrary groups, in
particular, we will discuss complete reducibility of representations of finite
groups and of compact groups.
\end{abstract}

\tableofcontents

\newpage



\section{Introduction}

Let $f : [-\pi,\pi] \to \R$ be a $2\pi$ periodic function, we may associate to it with a
sequence \emph{Fourier coefficients} $\hat{f} : \Z \to \R$, defined by:
\[
  \hat{f}(n) \triangleq \frac{1}{2\pi} \int_{-\infty}^{\infty} f(\theta) e^{-in\theta} d\theta
\]
and approximate the original function by the \emph{Fourier series} of $f$:
\[
  f(\theta) \sim \sum_{-\infty}^{\infty} \hat{f}(n) e^{in\theta}
\]
If $f \in L^2(-\pi, \pi)$, by Theorem 8.43 in \cite{folland}, we can establish a
correspondence between the series and $f$ by showing the series converges to $f$
in norm (of the $L^2$ space), that is,
\[
  \lim_{N \to \infty} \int_{-\pi}^{\pi} \abs{f(\theta) - \sum_{-N}^{N} \hat{f}(n)
    e^{in\theta}} = 0
\]
Moreover, since the integral of $e^{in\theta}$ from $\theta = -\pi$ to $\pi$ is zero
except for $n = 0$, $\langle e^{in\theta}, e^{im\theta} \rangle = 0$ for all $m
\neq n$. Hence $\{e^{in\theta}\}_{-\infty}^{\infty}$ forms an \emph{orthogonal
  basis} for $L^2(-\pi,\pi)$.

The first generalization one may make is to allow $f$ to be a (possibly)
non periodic function from $\R$ to $\R$, and consider the ``continuous''
analogue of Fourier coefficients.  This generalization defines us the
\emph{Fourier transform} of $f$, $\mathcal{F}f = \hat{f} : \R \to \R$:
\[
  \hat{f}(\xi) \triangleq \frac{1}{2\pi} \int_{-\infty}^{\infty} f(\theta)
  e^{-i\xi\theta} d\theta
\]
and the corresponding approximation for the original $f$:
\begin{equation}
  \label{eq:fourier-inversion}
  f(\theta) \sim \int_{-\infty}^{\infty} \hat{f}(\xi) e^{i\xi\theta} d\xi
\end{equation}
If we restrict $f \in L^2(\R)$, then by Theorem 1.9 in Chapter 5 of
\cite{stein2011fourier}, \cref{eq:fourier-inversion} turns out to be an
equality, hence the name: \emph{Fourier inversion} formula. By the same argument
as before, the set $\{e^{i\xi\theta}\}$ is an orthogonal basis of $L^2(\R)$.

The second and the third possible generalization follows from the observation
that the Fourier transform $\mathcal{F}(-)$ is a endomorphism on the group
$L^2(\R)$ under convolution.  If we define an operator $T_t : L^2(\R) \to
L^2(\R)$, $(T_tf)(\theta) \triangleq f(\theta + t)$ for every $t \in \R$, and
define the group action $\pi : \R \times L^2(\R) \to L^2(\R)$ by $\pi(t)f =
\mathcal{F}(T_tf)$. A simple calculation by substitution shows, for a given $t
\in \R$,
\[
  \widehat{T_t f}(\xi)
  = \frac{1}{2\pi} \int_{-\infty}^{\infty} f(\theta + t) e^{-i\xi\theta} d\theta
  = \frac{1}{2\pi} \int_{-\infty}^{\infty} f(\theta) e^{-i\xi(\theta - t)} d\theta
  = \hat{f}(\xi) \cdot e^{it\xi}
\]
Hence $(L^2(\R), \pi)$ is a representation of $\R$ in the inner product space
$L^2(\R)$ (these terms will be defined more precisely in Section 3).

If we replace $\R$ by a finite group $G$ and the inner product space $L^2(\R)$
by a generic finite-dimensional vector space $V$, we obtain what's called
\emph{discrete Fourier transform}.  Given this generalization, we are also
interested in questions arose classic Fourier theory, that is, (1) whether it is
possible to find a decomposition of elements in $V$ into linear combinations of
a basis and (2) whether an orthogonal basis exist.

The main portion of the paper will be devoted to investigate the same question
for the case where $G$ is an non-abelian compact group and the vector space $V$
is a Hilbert space.

% And the third
% generalization 


% the
% function space $L^2(\R)$ is a group representation of the group additive group
% $\R$ with the Fourier transform $\mathcal{F}(-)$ being the group action.



% The purpose of this paper is to discuss Peter-Weyl's Theorem, a classical result
% in representation theory that generalizes the theory Fourier Transformation.  The
% theorem and machineries all follows from the observation that $L^2(\T)$, the
% space of square integrable functions on the circle group, serves as a group
% representation of $\T$.

% Peter-Weyl's Theorem, when instantiated with group $\T$, asserts the
% representation $L^2(\T)$ may be decomposed into a direct sum of
% finite-dimensional irreducible representations. Moreover, the matrix
% coefficients of the irreducible unitary representations form an orthonormal
% basis of $L^2(\T)$, namely $\{e^{2\pi i n x}\}_{n \in \mathbb{Z}}$.

In Section 2, we will develop background machineries from measure theory and
functional analysis that allow us to pull off the main result in Section 4.  In
Section 3, we will investigate the second generalization of abstract Fourier
analysis, namely discrete Fourier analysis, and in Section 4 we will establish
the result of abstract Fourier analysis on compact groups.

% We will start by introducing background definitions and developing results in
% representation theory that allow us to discuss abstract ``Fourier Analysis'' as
% group actions on arbitrary vector spaces.  In Section 2, we discuss such
% representations on finite groups and we will extend those theories to
% representations of c
% ompact groups in Section 3.  Following that is a proof of
% Peter-Weyl's Theorem in Section 4.  Finally, we will discuss a categorification of
% the abstraction process we've gone through and show the equivalence between
% category of representations of $G$ and category of vector spaces parametrized by
% the dual group $\hat{G}$ of characters.

% % At the end of the paper, we will also discuss a categorification of the result
% % that establishes an equivalence between the category of representations of $G$
% % and the category of 


% \newpage
\section{Preliminaries}

Let $G$ be a group, that is, an underlying set $\mathbb{G}$ with operations that
give $\mathbb{G}$ the group structure, if we further enforce a topological
structure on $\mathbb{G}$, then we can treat it as a topological space given
those algebraic operations are continuous. We call $G$ a \emph{topological group}.

Some examples of topological groups include the group $\R^{\times}$ of real
numbers under multiplication: its' topology was induced from the metric $d(x,y)
= \abs{x - y}$.  Since the topology of $\R^{\times}$ is the usual one induced by
the metric, by $\epsilon-\delta$ argument, it's easy to show the map $(x,y)
\mapsto x \times y$ from $\R^2$ to $\R$ is continuous.

Another example of a topological group that is closely related to Fourier
analysis is the circle group $\T = \SetForm{z \in \C}{\abs{z} = 1}$, under
multiplication.  Equivalently, we can represent elements of $\T$ in polar
coordinates: $\T = \SetForm{e^{i\theta}}{\theta \in [-\pi,\pi]}$.  The
multiplicative identity is $e^{0} = 1$, and inverse operations is simply
$\left(e^{i\theta} \right)^{-1} = e^{-i\theta}$.

Apparently, $\T$ is not a finite group since the interval $[-\pi, \pi]$ is
uncountable, it's a topological group where the topological structure is
inherited from the usual topological structure of $\C$. Furthermore, $\T$ is a
compact group since we know the unit circle is compact in $\C$.

\subsection{Measure Theory Backgrounds}

Let $X$ be a topological space, we say $X$ is a \emph{measure space} if there
exist a $\sigma$-algebra $\mathfrak{M}$ in $X$, that is, a family of subsets of
$X$ containing $X$ and is closed under taking complements and countable union.
We call elements of $\mathfrak{M}$ \emph{measurable sets}. Let $f : X \to Y$ be
a map from a measurable space $X$ to a topological space $Y$ \emph{measurable}
if for every open set $V$ in $Y$, the pullback $f^{-1}(V)$ is a measurable set.

Let $\mathcal{T}$ be a topology on $X$, let $\Omega$ be a family of all
$\sigma$-algebras in $X$ that contains $\mathcal{T}$.  Then by intersecting all
elements in this family, we have a \emph{smallest} $\sigma$-algebra
$\mathcal{B}$ that contains all open sets of $X$, we refer to elements of
$\mathcal{B}$ as \emph{Borel sets}.

Given $(X,\mathfrak{M})$ a measure space, we define a \emph{measure} to be a map
$\mu : \mathfrak{M} \to [0,\infty]$ that is countably additive. This means, for
any disjoint countable collection $\{A_j\} \subset \mathfrak{M}$,
\[
  \mu\left( \bigcup_{j = 0}^{\infty} A_j \right) = \sum_{j = 0}^{\infty} \mu(A_j)
\]
If the $\sigma$-algebra above is the collection of Borel sets of $X$, then we
call the measure a \emph{Borel} measure.  We say the Borel measure is
\emph{regular} if it satisfies, for all measurable set $X$,
\[
  \mu(X) = \inf \SetForm{\mu(U)}{U \supset X, U \text{ open}} = \sup
  \SetForm{\mu(K)}{K \subset X, K \text{ compact}}
\]
If the topological space $G$ is a \emph{locally compact} group (i.e. every point
of $G$ has a compact neighborhood), there is, up to a constant multiple, a
unique regular Borel measure $\mu_L$ that is invariant under left translation.
By \emph{left translation invariance}, we mean $\mu_L(X) = \mu_L(gX)$ for all
measurable sets $X$ and $g \in G$.  We call such measure a \emph{left Haar
  measure}, it should also satisfy $\mu_L(X) < \infty$ for $X$ compact and
$\mu_L(X) > 0$ for all $X$ measurable. (For the existence and uniqueness of
$\mu_L$, see \cite{halmos2013measure}).  Similarly, we can also define
\emph{right Haar measure}.  Usually left and right Haar measure do not coincide
for general topological groups, when they do, we call the group
\emph{unimodular}.

One of the main purpose of inventing the formalism of measure is to develop the
theory of integration with respect to a given measure.  The idea of computing
the integral $\int_X f(x) d\mu(x)$ is to approximate the function $f$ by
\emph{simple functions}, that is, a linear combinations of characteristic
function on measurable sets.  If $f : X \to \R$ is non-negative, define
\[
  \int_X f(x) d\mu(x) \triangleq \sup_{0 \leq s \leq f} \int_X s(x) d\mu(s)
  = \sup_{0 \leq \sum \lambda_i \chi_{E_i} \leq f} \int_X \sum_{i = 0}^{n} \lambda_i \chi_{E_i}(x) d\mu(x)
\]
% Similar to the finite case, a representation of $G$ is a vector space $V$
% associated with a group action $\pi$.  If $V$ is finite-dimensional, then the
% action $\pi : G \to GL(V)$ is a continuous homomorphism where the topology of
% $GL(V)$ is inherited from the space $End(V)$ of linear self-maps.  In this case,
% we can define \emph{characters} similarly as a complex valued function
% $\chi_{\pi}(g) = \Tr \pi(g)$, we say a character is \emph{irreducible} if it's a
% character of a irreducible representation (i.e. contains no proper nonzero invariant spaces).

% On the other hand, if $V$ is infinite dimensional, we instead say $V$ is
% irreducible if it has no proper nonzero invariant \emph{closed} subspaces.

Notice, if $\mu_L$ is a left Haar measure, then for $\gamma \in G$,
\[
  \int_G f(\gamma g) d \mu_L(g) = \int_G f(\gamma g) d\mu_L(\gamma g)
\]
substituting $h = \gamma g$, we have
\[
  \int_G f(\gamma g) d \mu_L(g) = \int_G f(g) d\mu_L(g)
\]

Similarly, to check $\mu_R$ is a right Haar measure, it amounts to check the
integral
\[
  \int_G f(g \gamma) d \mu_L(g) \stackrel{?}{=} \int_G f(g) d \mu_L(g)
\]

Consider the conjugation map $\phi(g) : h \mapsto g^{-1} h g$, it's a
automorphism on $G$ (with inverse being $\phi(g)^{-1} : h \mapsto g h g^{-1}$).
Since left Haar measures are unique up to a constant multiple, the left Haar
measure of $\phi_g(G)$ is a constant multiple of the left Haar measure of $G$.
In another words, there exist a function $\delta : G \to \R_+^{\times}$ such that
\[
  \int_G f(\phi_g(h)) d\mu_L(h) = \delta(g) \int_G f(h) d\mu_L(h)
\]

Given $g_1, g_2 \in G$, since conjugating first by $g_1$ then by $g_2$ is
equivalent to conjugating by $(g_2g_1)$, we have $\delta(g_2) \cdot \delta(g_1)
= \delta(g_2g_1)$, the map $\delta$ is a group homomorphism.
% Generally, we call a homomorphism from $G$ to $\C^{\times}$ a \emph{quasicharacter}.
% Then composing
% it in the integral $\int_G f(\phi g) d\mu_L(g)$ should given a constant multiple
% of $\int_G f(g) d\mu_L(g)$.

\begin{lem}
  Let $\mu_L$ be a left Haar measure of $G$, let $\delta : G \to \R_+^{\times}$
  be defined as above, then $\delta \cdot \mu_L$ is a right Haar measure of $G$.
\end{lem}

\begin{proof}
  We verify the measure $\delta(h)\mu_L(h)$ is right invariant.  Let
  $\widetilde{f} = f \cdot \delta$, then by definition of $\delta$,
  \[
    \int_G \widetilde{f}(g^{-1} h g) d\mu_L(h) =
    \delta(g) \int_G \widetilde{f}(h) d\mu_L(h)
  \]
  Since $\mu_L$ is left invariant, left translate $\widetilde{f}$ by $g$ gives
  \[
    \int_G \widetilde{f}(hg) d\mu_L(h) = \delta(g) \int_G \widetilde{f}(h) d\mu_L(h)
  \]
  Since $\delta$ is a homomorphism, we expand the left hand side:
  \[
    \int_G f(hg) \delta(h) \delta(g) d\mu_L(h) = \delta(g) \int_G \widetilde{f}(h) d\mu_L(h)
  \]
  Canceling $\delta(g)$ from both sides, we have
  \[
    \int_G f(hg) \delta(h) d\mu_L(h) = \int_G \widetilde{f}(h) d\mu_L(h) =
    \int_G f(h) \delta(h) d\mu_L(h)
  \]
  Thus, $\delta \cdot \mu_L$ is a right Haar measure.
\end{proof}

The property of Haar measure on compact groups follows from the Lemma above:

\begin{lem}
  Let $G$ be a compact group, then
  \begin{enumerate}
    \item[(1)] $G$ is unimodular
    \item[(2)] $\mu_L(G) < \infty$
  \end{enumerate}
\end{lem}

\begin{proof}
  Let $\delta : G \to \R_+^{\times}$ be define as above.  Since $G$ is compact
  and $\delta$ is continuous, the image $\Ima(\delta)$ is a compact subgroup of
  $\R_+^{\times}$.  Since the only compact subgroup of $\R_+^{\times}$ is
  trivial, the image is trivial.  Then the right Haar measure $\delta \cdot
  \mu_L$ is equal to $\mu_L$, hence $G$ is unimodular.  Since $G$ is compact, by
  assumption for Haar measure, $\mu(G) < \infty$.
\end{proof}

Since $\mu_L(G) < \infty$ and left Haar measures (also right Haar measure when
$G$ is compact) are unique upto a constant multiple, we can pick a measure $\mu$
such that the integral is normalized, that is, $\int_G 1 d\mu(g) = 1$. Without
loss of generality, we will always consider Haar measures that normalizes the
integral in later sections. For notational convenience, later in the paper we
will use $dg$ in place for $d\mu(g)$.

\subsection{Functional Analysis Machineries}

In order to discuss whether we have ``enough'' matrix coefficients, we will
first need to develop some results on linear operators, in particular
on \emph{compact operators}.

Let $\mathfrak{H}$ be a normed vector space with norm $\norm{-}$, a linear
operator $T : \mathfrak{H} \to \mathfrak{H}$ (i.e. maps compatible with the
linear structure of $\mathfrak{H}$) is called \emph{bounded} if there exist a
constant $C$ such that
\[
  \norm{Tx} \leq C\norm{x} \qquad (\forall x \in \mathfrak{H})
\]
The smallest such constant is referred to as the \emph{operator norm} of $T$,
denote by $\abs{T}$.  It's easy to see $T$ is bounded iff $T$ is continuous (on
the metric topology given by the norm).

We call the operator $T$ \emph{self-adjoint} if
\[
  \langle T f, g \rangle = \langle f, T g \rangle \qquad (\forall f,g \in \mathfrak{H})
\]

A bounded operator $T : \mathfrak{H} \to \mathfrak{H}$ is \emph{compact} if
given any bounded sequence $\{x_1, x_2, \ldots\} \subset \mathfrak{H}$, the
image sequence $\{Tx_1, Tx_2, \ldots\}$ has a convergent subsequence.

For $f,g \in \mathfrak{H}$, we call $f$ an \emph{eigenvector} with
\emph{eigenvalue} $\lambda$ of $T$ if $f \neq 0$ and $Tf = \lambda f$. As usual,
we call the space $\SetForm{f \in \mathfrak{H}}{f \neq 0, Tf = \lambda f} \cup \{0\}$ the
$\lambda$-eigenspace.

\begin{thm}[Spectral theorem for compact operators] \label{spectral}
  Let $T$ be a compact self-adjoint operator on a Hilbert space $\mathfrak{H}$.
  Let $\mathfrak{N}$ be the nullspace of $T$. Then
  \begin{enumerate}
    \item[(1)] the Hilbert space dimension of $\mathfrak{N}^{\perp}$ is at most countable.
    \item[(2)] $\mathfrak{N}^{\perp}$  has an orthonormal basis
      $\{\phi_i\}_1^{\infty}$ of eigenvectors of $T$, and associated set of
      eigenvalues $\{\lambda_i\}_1^{\infty}$.
    \item[(3)] If $\mathfrak{N}^{\perp}$ is not finite-dimensional, the
      eigenvalues $\lambda_i \to 0$ as $i \to \infty$.
  \end{enumerate}
\end{thm}

\begin{proof}
  See Theorem 4.11-1 in \cite{ciarlet2013linear}.
\end{proof}

Notice, if $\mathfrak{N}^{\perp}$ is not finite-dimensional, then each
$\lambda_i$ may only appear finitely many times in the sequence of eigenvalues,
hence there are only finitely many eigenvectors associated with $\lambda_i$, the
$\lambda_i$-eigenspace is finite.

Let $X$, $Y$ be compact topological spaces, $Y$ is has a metric $d$.  Let $U$ be
a set of continuous maps $X \to Y$, we call $U$ being \emph{equicontinuous} if
for every $x \in X$ and $\epsilon > 0$, there exists a neighborhood $N \ni x$
such that
\[
  d(f(x), f(x')) < \epsilon \qquad (\forall x' \in N, f \in U)
\]

The following theorem relates equicontinuity with compactness:

\begin{thm}[Ascoli and Arzela] \label{ascoli-arzela}
  Let $X$ be a compact space, $U \subset C(X)$ is a bounded subset of continuous
  functions, with the sup-norm metric. Suppose $U$ is equicontinuous, then every
  sequence in $U$ has a uniformly convergent subsequence.
\end{thm}

\begin{proof}
  See Theorem 3.10-1 in \cite{ciarlet2013linear}.
\end{proof}
% \newpage 

\section{Representation Theory for Finite Groups}

\begin{defn}[Group action]
  Let $G$ be a group, the action of $G$ on set $X$ is a map $\alpha : G \times X
  \to X$ that is compatible with the group structure.  That means, for all $g,h
  \in G$, $x \in X$,
  \begin{align*}
    \alpha(gh)x & = \alpha(g)(\alpha(h)x)\\
    \alpha(1_G)x & = x
  \end{align*}
\end{defn}

\begin{defn}[Linear Representation]
  Let $G$ be a group, a \emph{linear representation} $\rho$ of $G$ on a complex
  vector space $V$ is a group action on $V$ that preserves the vector space
  structure of $V$.
\end{defn}

If the context is clear, we would simply refer to a representation $(\rho, V)$
of group $G$ by the vector space $V$.  We will also write $gv$ in place for
$\rho(g)(v)$ when the $G$-action $\rho$ is clear from the context.
% Throughout
% this paper, we will always be interested in complex representations, however,
% the following Proposition is true generally:

\begin{prop}
  The set of representations of $G$ on $V$ is in bijection with the set of group
  homomorphisms from $G$ to $GL(V)$.
\end{prop}

\begin{proof}
  Let $(\rho, V)$ be a $G$-representation.  By definition of group action,
  $\rho$ preserves group structure, in particular $\rho(g)$ is invertible for
  every $g \in G$. Conversely, let $f \in Hom(G,GL(V))$ be given, then for every
  $g \in G$, $f(g) \in GL(V)$, thus preserves $V$ structure.  Furthermore, since
  $f$ is a homomorphism, it preserves $G$ structure, hence a $G$-representation.
\end{proof}

\begin{defn}[Isomorphic class of reps]
  Let $G$ be a group, $(\rho_1, V_1)$ and $(\rho_2, V_2)$ be two representations
  of $G$.  An \emph{isomorphism} between those two representations is a linear
  isomorphism $\phi : V_1 \to V_2$ such that the following diagrams commutes for
  every $g \in G$:
  \[
    \begin{tikzcd}
      V_1 \arrow[d, "\phi", "\vsimeq"'] \arrow[r, "\rho_1(g)"] & V_1 \arrow[d, "\vsimeq"', "\phi"]\\
      V_2 \arrow[r, "\rho_2(g)"] & V_2
    \end{tikzcd}
  \]
  In this case, we asy $V_1$ and $V_2$ are \emph{equivalent representations}.

  An \emph{isomorphic class of representations} of $G$ is thus defined to be a
  equivalence class of representations that are equivalent to each other.


  Generally, if $\phi$ is not an isomorphism, it's referred to as an
  \emph{interwining operator} or a \emph{$G$-linear map}.  For notational
  convenience, let $\Hom^G(V,W)$ denote the space of $G$-linear maps.
\end{defn}

\subsection{Representation Constructions}

% In this subsection, we will define some basic constructions of representations
% and introduce some canonical examples of representations.

Since the definition $G$-representation relies on a linear structure, a number
of constructions of representations are inherited from constructions on vector
spaces.

\begin{defn}[Direct sum of representations]
  Let $(\rho_1, V_1)$ and $(\rho_2, V_2)$ be two representations of group $G$,
  the \emph{direct sum} of two representations is the space $V_1 \oplus V_2$
  equipped with action $\rho_1 \oplus \rho_2 = \alpha \circ (\rho_1 \times
  \rho_2)$.  Where $\alpha : GL(V_1) \times GL(V_2) \to GL(V_1 \oplus V_2)$
  is a map obtained by coordinate-wise action.

  Notice the map $\rho_1 \oplus \rho_2$ may also be identified by the block
  diagonal matrix:
  \[
    \rho_1 \oplus \rho_2 : g \mapsto
    \begin{pmatrix}
      \rho_1(g) & 0\\
      0      & \rho_2(g)
    \end{pmatrix}
  \]
\end{defn}

\begin{defn}[Dual representation]
  Let $(\rho, V)$ be a representation of group $G$, the \emph{dual
    representation} is the dual space $V^* = \Hom(V,k)$ with action $\hat{\rho} : G
  \times V^* \to V^*$:
  \[
    \hat{\rho}(g) : L \mapsto L \circ \rho(g)^{-1}
  \]
  we call $\hat{\rho}$ the \emph{contragradient} of $\rho$.  It's easy to check
  \begin{align*}
    \hat{\rho}(g_1g_2) & = (L \mapsto L \circ \rho((g_1g_2^{-1})))\\
                       & = (L \mapsto L \circ \rho(g_2^{-1}g_1^{-1}))\\
                       & = (L \mapsto L \circ \rho(g_2^{-1}) \circ \rho(g_1^{-1}))\\
                       & = \hat{\rho}(g_1) \circ (L \mapsto L \circ \rho(g_2^{-1}))\\
                       & = \hat{\rho}(g_1) \circ \hat{\rho}(g_2)
  \end{align*}
\end{defn}

\begin{defn}[Subrepresentation]
  Let $(\rho, V)$ be a representation of group $G$. A \emph{subrepresentation}
  of $V$ is a $G$-invariant subspace $W \subset V$, that is, $W$ satisfies
  \[
    \rho(g)(\bm{w}) \in W \qquad (\forall \bm{w} \in W, \forall g \in G)
  \]
  and $W$ is a representation of $G$ under the map $\rho$.
\end{defn}

\begin{defn}[Reducibility]
  A $G$-representation $(\rho, V)$ is \emph{irreducible} if it contains no
  proper invariant subspaces.  It's \emph{completely reducible} if it may be
  decomposed into a direct sum of irreducible representations.
\end{defn}

\begin{defn}[Quotient representation]
  Let $(\rho, V)$ be a representation of group $G$, $W$ is a subspace of $V$.
  The action map of the \emph{quotient representation} is defined through the
  action map of the original representation, $\widetilde{\rho} : G \times V/W \to V/W$:
  \[
    \widetilde{\rho}(g)(\bm{v} + W) \triangleq \rho(g)(\bm{v}) + W
  \]
\end{defn}

\subsection{Complete Reducibility for Finite Groups}

In this section, we establishes the reducibility theorem of representations of
finite groups. The main result is the theorem that every $\C$ representation
admits a unique decomposition into irreducible representations.

\begin{defn}[Unitary representation]
  Let $G$ be a group, a representation $(\rho,V)$ is \emph{unitary} if there
  exist a positive definite Hermitian inner product $H$ that is invariant under
  $G$-actions, that is,
  \[
    H(v,w) = H(gw, gw) \qquad (\forall v,w \in V, \forall g \in G)
  \]

  A representation is called \emph{unitarisable} if it can be equipped with such
  an inner product.
\end{defn}

\begin{lem}\label{ortho-inv}
  Let $V$ be a unitary representation of finite group $G$, $W$ is an invariant
  subspace. Then $W^{\perp}$ is also an invariant subspace.
\end{lem}

\begin{proof}
  Let $w' \in W^{\bot}$, $g \in G$ be given.  Let $H$ be the inner product on
  $V$ that is invariant under $G$-actions (we call such an inner product
  \emph{$G$-equivariant}).  Then, we have,
  \begin{align*}
    & H(w,w') = 0 && \text{definition of orthogonal set}\\
    \Leftrightarrow \ & H(gw, gw') = 0 && \text{invariance}\\
    \Leftrightarrow \ & H(w, gw') = 0  && g^{-1}gw = w \in W\\
    \Leftrightarrow \ & gw' \in W^{\bot} && \text{definition}
  \end{align*}
  Thus, $W^{\bot}$ is an invariant subspace.
\end{proof}

\begin{lem}\label{fin-decompose}
  Let $G$ be a finite group, $(\rho, V)$ be a representation of $G$ ($\dim V <
  \infty$), $W$ be a subrepresentation of $V$.  Then, there exist a
  complementary invariant subspace $W'$ of $V$ such that $V = W \oplus W'$.
\end{lem}

\begin{proof}
  We will prove the theorem by using inner products and Weyl's technique of
  averaging over the group.

  Let $H_0$ be an arbitrary Hermitian inner product of $V$, we construct a
  Hermitian inner product that is invariant under $G$-actions as follows:
  \[
    H(v,w) \triangleq \sum_{g \in G} H_0(gv, gw)
  \]
  We verify the invariance: let $g \in G$ be given, we would like to show
  \[
    H(gv, gw) = \sum_{g' \in G} H_0(g'gv, g'gw) \stackrel{?}{=} \sum_{h \in G}
    H(hv, hw) = H(v,w)
  \]
  It suffices to show, for any $g \in G$, there exists a bijection: $Gg \cong
  G$.  From left to the right, let $g' \in G$ be given, we simply put $h = g'g$.
  From right to left, let $h \in G$ be given, we define $g' = hg^{-1}$.

  By \ref{ortho-inv}, since $V$ is unitary, $W^{\bot}$ is an invariant subspace
  and orthogonal decomposition $V = W \oplus W^{\bot}$.
\end{proof}

\begin{cor}[Complete reducibility]\label{fin-decompose-exist}
  Any representation of a finite group (hence must be finite dimensional) admits
  an orthogonal decomposition into irreducible sub-representations.
\end{cor}

\begin{proof}
  Finite decomposition using \ref{fin-decompose}.
\end{proof}

Now, we would like to show that such a decomposition is unique.

\begin{lem}\label{im-ker-inv}
  Let $V$, $W$ be representations of a finite group $G$, $\phi \in \Hom^G(V,W)$, then
  \begin{enumerate}
    \item
      $\Ima \phi$ is an invariant subspace of $W$
    \item
      $\ker \phi$ is an invariant subspace of $V$.
  \end{enumerate}
\end{lem}

\begin{proof}\
  \begin{enumerate}
    \item
      Let $w \in \Ima \phi$ be given, then there exist $v \in V$ such that $\phi(w)
      = w$.  Since $\phi$ commutes with group actions,
      \[
        \phi(gv) = g\phi(v) = gw
      \]
      Thus, $gw \in \Ima \phi$.

    \item
      Let $v \in \ker \phi$ be given, then $\phi(v) = 0 \in W$. Since $\phi$
      commutes with group actions, and group actions preserves vector space structure,
      \[
        \phi(gv) = g\phi(v) = g0 = 0
      \]
      Thus $gv \in \ker \phi$.
    \end{enumerate}
\end{proof}

\begin{lem}[Schur's Lemma]
  Let $V$, $W$ be irreducible representations of finite group $G$, $\phi \in \Hom^G(V,W)$.
  Then
  \begin{enumerate}
    \item Either $\phi$ is an isomorphism or $\phi \equiv \bm{0}$.
    \item If $V = W$, then $\phi = \lambda \cdot I$ for some $\lambda \in
      \C$, where $I$ is the identity map.
  \end{enumerate}
\end{lem}

\begin{proof}\
  \begin{enumerate}
    \item Follows from \ref{im-ker-inv}.
    \item Since $\C$ is algebraically closed, $\phi$ has at least one
      eigenvalue $\lambda$, let $\bm{v} \neq 0$ be the associated eigenvector, then
      $\phi \bm{v} = \lambda \bm{v} \Rightarrow (\phi - \lambda \cdot I)\bm{v} =
      \bm{0}$, hence $(\phi - \lambda \cdot I)$ is not injective, by (1), it
      must be $0$, thus $\phi = \lambda \cdot I$.
  \end{enumerate}
\end{proof}

\begin{thm}(Uniqueness of decomposition)
  Any representation $(\rho,V)$ of a finite group $G$ admits a unique orthogonal
  decomposition into irreducible subrepresentations.
  \[
    V = \bigoplus V_j^{\oplus e_j}
  \]
  or equivalently, we may express the decomposition through the action
  \[
    \rho = \sum_{j} m_j \rho_j
  \]
\end{thm}

\begin{proof}
  Existence is given by \ref{fin-decompose-exist}.
  Suppose $V = \bigoplus W_k^{\oplus f_k}$ is another decomposition, then
  consider the identity map $\bm{1} : V \to V$ that maps $\bigoplus V_j^{\oplus
    e_j}$ to $\bigoplus W_k^{\oplus f_k}$, then by Schur's Lemma, it must be the
  case that $\bm{1}$ maps $V_j^{\oplus e_j}$ into $W_k^{\oplus f_k}$ where $V_j$
  and $W_j$ are isomorphic and $e_j = f_k$.  This proves the uniqueness.
\end{proof}


% \subsection{Character Theory for Finite Groups}

% In this section, we discuss a very important numerical invariant of
% representations of groups.  They are important in the sense that they
% essentially determines representations they are associated to.

% We will first develop results of character theory with finite groups, and in
% latter sections, we will show how analogous definitions and theorems may be
% obtained with compact groups.  In this section, we only discuss representations
% on field $\C$, and $G$ will denote a finite group.

% \begin{defn}[Character]
%   Let $(\rho,V)$ be a representation of $G$, it's character $\chi_{V}$ is
%   the complex-valued function of $G$ defined by
%   \[
%     \chi_V(g) \triangleq \Tr_V(\rho(g))
%   \]
%   where $\Tr_V(\rho(g))$ is the trace of $\rho(g) \in GL(V)$ on $V$.  By
%   properties of trace, we have $\chi_V(hgh^{-1}) = \chi_V(g)$. Thus the
%   character is constant on the conjugacy classes of $G$, such a function is
%   called a \emph{class function}.
% \end{defn}

% The definition above gives a number of nice properties about characters:

% \begin{prop}
%   Let $V$ be a representation of $G$. Then
%   \begin{enumerate}
%     \item[(1)] $\chi_V(1) = \dim V$
%     \item[(2)] $\chi_V(g^{-1}) = \conj{\chi_V(g)}$
%     \item[(3)] $\chi_{V^*}(g) = \chi_V(g^{-1})$
%   \end{enumerate}
% \end{prop}

% \begin{prop}\label{char-prop}
%   Let $V$, $W$ be representations of $G$. Then
%   \[
%     \chi_{V \oplus W} = \chi_{V} + \chi_{W},
%     \quad
%     \chi_{V \otimes W} = \chi_{V} \cdot \chi_{W}
%   \]
% \end{prop}

% \begin{thm}
%   A general statement of \ref{char-prop} is: the character is a ring
%   homomorphism from $R_G$ to the class functions on $G$. It takes the involution
%   $V \leftrightarrows V^*$ to complex conjugation.
% \end{thm}

% \begin{proof}
%   To be elaborated.
% \end{proof}

% Let $\mathcal{F}(G)$ denote the space of function on group $G$, we can define an
% Hermitian inner product on this space: Given $\phi, \psi \in \mathcal{F}(G)$,
% \[
%   H(\phi,\psi) \triangleq \frac{1}{\abs{G}} \sum_{g \in G} \conj{\phi(g)} \cdot \psi(g)
% \]
% If we sum over conjugacy classes of $G$, we obtain an inner product of the space
% of class functions on $G$:
% \[
%   H(\phi,\psi) \triangleq \frac{1}{\abs{G}} \sum_{C \subset G} \abs{C} \cdot
%   \conj{\phi(C)} \psi(C)
% \]

% \begin{lem}
%   Let $(\rho, V)$, $(\pi, W)$ be irreducible representations of $G$, let $T \in
%   \Hom^G(V, W)$, define
%   \[
%     \widetilde{T} = \frac{1}{\abs{G}} \sum_{g \in G} \pi(g) \circ \phi \circ \rho(g)^{-1}
%   \]
%   then,
%   \begin{enumerate}
%     \item[(1)] $\widetilde{T} \in \Hom^G(V,W)$.
%     \item[(2)] If $V = W$, then $\Tr T = \Tr \widetilde{T}$.
%   \end{enumerate}
% \end{lem}

% \begin{proof}
% \end{proof}

% \begin{thm}[Orthogonality of characters]
%   Let $V$, $W$ be irreducible representations of $G$, then If $V$, $W$ are
%   equivalent, then $H(\chi_V, \chi_W) = 1$, else $H(\chi_V, \chi_W) = 0$.
% \end{thm}

% \begin{proof}
% \end{proof}

% \begin{cor}
%   Let $(\rho, V)$ be a representation of $G$, and it admits a decomposition
%   $\rho = \sum_{j} m_j \rho_j$ then, $m_j = H(\chi_V, \chi_{V_j})$.
% \end{cor}

% \begin{cor}
%   A representation $(\rho, V)$ is irreducible iff $\norm{\chi_{V}}^2 = 1$.
% \end{cor}

% \begin{cor}
%   Two representations are equivalent iff they have the same character.
% \end{cor}
% \newpage

\section{Compact Groups}

% In previous sections, we have shown the existence and the uniqueness of
% isotypic decomposition of representations of finite groups.  However, the major
% theorem we would like to prove was about the similar decomposition but for
% general compact groups.

% In the following section, we will introduce notions of compact groups, and
% properties of representations on them: can they be decomposed, is the
% decomposition is isotypic?
% explore the classical development of
% representation theory of compact groups.

% \begin{defn}[Topology]
%   A collection $\tau$ of subsets of set $X$ is said to be a \emph{topology} of
%   $X$ if it satisfies:
%   \begin{enumerate}
%     \item[(i)] $\varnothing \in \tau$, $X \in \tau$
%     \item[(ii)] If $V_i \in \tau$ for $i = 1, \ldots , n$, then $\bigcap_{i = 1}^{n} V_i \in \tau$
%     \item[(iii)] If $\{V_{\alpha}\}_{\alpha \in I}$ in an arbitrary collection
%       of subsets of $\tau$, then $\bigcup_{\alpha \in I} V_{\alpha} \in \tau$.
%   \end{enumerate}

%   If $\tau$ is a topology of $X$, then $X$ is called a \emph{topological space},
%   whose open sets are elements in $\tau$.  $X$ is \emph{locally compact} if
%   every point in $X$ has a compact neighborhood.

%   A function $f : X \to Y$ between two topological spaces are said to be
%   \emph{continuous} if $f^{-1}(V)$ is open for every $V \subset Y$ open. $f$ is
%   said to be \emph{continuous at $x_0$} if to every neighborhood (open set) $V
%   \ni f(x_0)$, there exist a corresponding neighborhood $W \ni x_0$ such that
%   $f(W) \subset V$.
% \end{defn}

We would like to answer the similar question about complete reducibility on
general compact groups.  Is it possible to develop similar results on finite
reducibility for general compact group?  Is it possible for us to adapt proofs
in the finite case to compact groups?

We will provide a affirmative answer to the first question: we will be able to
show the reducibility of representation of a compact group into a orthonormal
set of irreducible representations, each of them is finite dimensional.
However, unfortunately we cannot adapt most of the proofs in the finite case to
the compact case (in fact, the only theorem we developed previously may be
carried over to the compact case is Schur's Lemma and complete reducibility of
\emph{finite} representations).
% We will need to develop a quite large
% toolbox of machineries that allow us to establish the result for arbitrary
% representations.

In short, the orthogonality will be established by Schur's orthogonality
theorem, and the complete reducibility will be established by Peter-Weyl's
Theorem. 


% \begin{exa}
%   Consider group
%   \[
%     G = \SetForm{\begin{pmatrix} y & x\\ 0 & 1 \end{pmatrix}}{x,y \in \R, y > 0}
%   \]
%   It has left- and right-invariant measures:
%   \[
%     d\mu_L = y^{-2} dx dy,
%     \quad
%     d\mu_R = y^{-1} dx dy
%   \]
% \end{exa}

\subsection{Complete Reducibility for Finite Representations}

Let $G$ be a compact group, similar to the finite case, we can define the notion
of a representation of $G$:  at the most general setting, a
\emph{representation} of $G$ is simply a group homomorphism $f : G \to H$, where
the group $H$ is not even necessarily associated with a linear structure.
However, in this paper, we will be only interested in \emph{linear
  representations}, thus $H = \End(V)$ for some vector space $V$.  However, in
contrast with the finite case, here $V$ may not have a finite basis.  When $V$
is infinite dimensional and is associated with a topology, we say $(\pi,V)$ is
\emph{irreducible} if it has no proper nonzero invariant \emph{closed}
subspaces.

In the following sections, we will be interested in general of $V$ being a
Hilbert space, that is, an inner product space which is complete under the norm
induced by the inner product. In particular, we will be interested in the
Hilbert space $L^2(G)$ of square integrable functions on $G$. The inner product
is given by conjugate integration (this is why we developed the theory of
integration on compact groups with Haar measures earlier):
\begin{equation} \label{Hilbert-inner-product}
  \langle f_1 , f_2 \rangle_{L^2} \triangleq \int_G f_1(g) \conj{f_2(g)} dg
\end{equation}
In addition, $L^2(G)$ is indeed a complete metric space with the metric induced
by the inner product defined above, for a detailed examination, see Theorem 3.11
in \cite{rudin1987real}.

In fact, similar to the case of finite group, in the general case of $G$ being a
compact group, we can show the existence of an $G$-equivariant inner product.
This also means, every finite dimensional representation of a compact group is
\emph{unitarisable}.

\begin{thm}
  Let $G$ be a compact group and $(\pi,V)$ be a finite dimensional
  representation of $G$, then $V$ admits a $G$-equivariant inner product.
\end{thm}

\begin{proof}
  We will again apply Weyl's trick of averaging over the group $G$, but this
  time by integrating over $G$: let $H_0(,)$ be an arbitrary inter
  product on $G$, define
  \[
    H(v,w) \triangleq \int_G H_0(\pi(g)v, \pi(g)w) dg
  \]
  Then, $H(,)$ is $G$-equivariant by construction.
\end{proof}

If we replace ``sum over group finite group $G$'' by ``integrate over compact
group $G$'' in \Cref{ortho-inv} and \Cref{fin-decompose}, etc. The result of
complete reducibility follows for \emph{finite} representations of compact group
$G$.

% \begin{thm}
%   Let $(\rho,V)$ be a finite dimensional unitary representation of compact group
%   $G$, $W$ is an invariant subspace of $V$. Then $W^{\perp}$ is also an
%   invariant subspace of $V$.
% \end{thm}

% \begin{thm}
%   Let $G$ be a compact group, $(\rho,V)$ be a finite dimensional representation
%   of $G$, $W$ be a subrepresentation of $V$.  Then there exist a complementary
%   invariant subspace $W'$ of $V$ such that $V = W \oplus W'$.
% \end{thm}

\begin{thm}[Complete reducibility for finite representations]
  Any finite representation of a compact group $G$ admits an orthogonal
  decomposition into irreducible sub-representations.
\end{thm}

Now, let's step to the complete reducibility of arbitrary representations.

\subsection{Matrix Coefficients}

Let $G$ be a compact group, $(\pi,V)$ be a representation of $G$.  Then if $V$
is finite dimensional, by choosing a basis for $V$, we can write out the matrix
representation of the linear map $\pi(g)$ for every $g \in G$.  Or equivalently,
we have a collection of functions $\phi_{ij} : G \to \C$ that tells us what is
the value of the $(i,j)$th matrix entry of $\pi(g)$.  Even more importantly, if
we are able to construct such a collection of functions, then we also obtain a
representation of $G$! The question is, how do we construct this collection of
functions?

Let's brute force through one possible construction. Under the given basis $B =
\{e_i\}_{i = 1}^{n}$, suppose we have a vector $v = (v_1, \ldots, v_n)^{T} =
\sum_1^n v_i e_i$ and an element $g \in G$. Suppose the matrix $\pi(g)$ is
represented by:
\[
  \begin{pmatrix}
    \phi_{11}(g) & \cdots & \phi_{1n}(g)\\
    \vdots & \ddots & \vdots\\
    \phi_{n1}(g) & \cdots & \phi_{nn}(g)
  \end{pmatrix}
\]

Computing $\phi(g)(v)$ gives, $\pi(g)(v) = (\sum_{1}^{n} \phi_{1i}(g) \cdot v_i,
\cdots, \sum_{1}^{n} \phi_{ni}(g) \cdot v_i)^{T}$, hence $\pi_{ij}(g) =
L_i(\pi(g)e_j)$, where $L_i$ is a linear functional on $V$ that picks out the
$i$-th component of the vector, $L_i(\sum a_j e_j) = a_i$.

This motivates us to define \emph{matrix coefficients}, an
abstract characterization of those family of functions independent of choices of
a basis or even a concrete vector space $V$.

\begin{defn}[Matrix Coefficients]
  Let $G$ be a group, $(\pi, V)$ be a finite dimensional representation of $G$,
  then \emph{matrix coefficients} on $G$ are maps of the form
  \[
    \phi : g \mapsto L(\pi(g)v)
  \]
  for $L \in V^*$, $v \in V$.
\end{defn}

Let $\mathcal{M}_{\pi}$ denote the set of matrix coefficients of representation
$(\pi,V)$, then it forms a vector space. Furthermore, we can show an intimate
relation between matrix coefficients and the dual representation.

\begin{prop} \label{dual-mat-coeff}
  $f$ is a matrix coefficient of $(\pi,V)$ iff $\check{f}(g) \triangleq
  f(g^{-1})$ is a matrix coefficient of the dual representation $(\hat{\pi},V^*)$.
\end{prop}

\begin{proof}
  By the identification $V \simeq V^{**}$, from left to right:
  \[
    \check{f}(g) = f(g^{-1}) = L(\pi(g^{-1})v) = (L \circ \pi(g)^{-1}) v = v^{**}(\hat{\pi}(g))
  \]
  From right to left:
  \[
    f(g) = \check{f}(g^{-1}) = v^{**}(\hat{\pi}(g^{-1})) = (L \circ \pi(g)) v = L(\pi(g)v)
  \]
\end{proof}

Since the vector space associated with matrix coefficients are
finite dimensional, by straightforward linear algebra, we may check the set of
linear functionals $B^* = \{L_i\}_{i = 1}^{n}$ induced from the basis $B =
\{e_i\}_{i = 1}^{n}$ for $V$ is a basis for the dual space $V^*$.  Hence
$\dim(\mathcal{M}_{\pi}) \leq \dim(V)^2$. This allows us to prove the following
theorem.

Let $\lambda(g)f$ and $\rho(g)f$ denote left and right translations by $g$ respectively,
\begin{equation}
  \label{eq:LR-translation}
  \lambda(g)(f) = x \mapsto f(g^{-1}x),
  \qquad
  \rho(g)(f) = x \mapsto f(xg)
\end{equation}

\begin{thm} \label{left-right-finite}
  Let $f$ be a function on $G$. Then the followings are equivalent:
  \begin{enumerate}
    \item $\lambda(g)f$ spans a finite dimensional vector space.
    \item $\rho(g)f$ spans a finite dimensional vector space.
    \item $f$ is a matrix coefficient of a finite dimensional representation.
  \end{enumerate}
\end{thm}

\begin{proof}
  Firstly, suppose (3), let $(\pi,V)$ be a finite dimensional representation of
  $G$ and $f(h) = L(\pi(h)v)$, then $(\lambda(g)(f))(h) = f(g^{-1}x) =
  L(\pi(g^{-1}x)v)$ and $(\rho(g)(f))(h) = f(xg) = L(\pi(xg)v)$, they are also
  matrix coefficients of $V$.  Because $\dim(V) < \infty$,
  $\dim(\mathcal{M}_{\pi}) \leq \dim(V)^2$, the vector spaces spanned by left
  and right translations also have finite dimension, hence $(3) \Rightarrow
  (1),(2)$.

  Suppose $\rho(g)(f)$ spans a vector space $V \subset \Hom(G,\C)$, $\dim(V) <
  \infty$, then $(\rho,V)$ is a representation of $G$: the action is $\rho(g)(v)
  = x \mapsto v(xg)$.  Define a linear functional $L \in V^* \subset
  \Hom(\Hom(G,\C),\C)$ by $L(\phi) = \phi(1)$, where $1$ is the unit of $G$.
  Then $L(\rho(g)f) = (\rho(g)(f))(1) = f(g1) = f(g)$, $f$ is indeed a matrix
  coefficient of $V$, hence $(2) \Rightarrow (3)$.
  
  Finally, suppose $\lambda(g)(f)$ spans a vector space $V \subset \Hom(G,\C)$,
  $\dim(V) < \infty$. Let $\check{f}(h) = f(h^{-1})$ as usual, and define
  $\widetilde{V}$ by
  \begin{align*}
    \widetilde{V} & = \SetForm{\lambda(g)f \circ (-)^{-1} \in \Hom(G,\C)}{g \in G}\\
    & = \SetForm{h \mapsto f(g^{-1}h^{-1})}{g \in G}
    = \SetForm{\rho(g)(\check{f})}{g \in G}
  \end{align*}

  Then $\dim(\widetilde{V}) < \infty$, by the previous argument, $\check{f}$ is
  a matrix coefficient, by \Cref{dual-mat-coeff}, $f$ is a matrix coefficient.
  Hence, $(1) \Rightarrow (3)$.
\end{proof}

\subsection{Schur's Orthogonality}

Before we dive into the discussion of orthogonality of matrix coefficients, we
firstly recall a remarkable theorem in functional analysis that uniquely
determines the forms of continuous linear functionals on a Hilbert space by its
inner product.

\begin{thm}[F. Riesz representation theorem in a Hilbert space] \label{riesz-rep-thm-hilbert}
  Let $(H,\langle \cdot, \cdot \rangle)$ be a Hilbert space over $\R$ or $\C$.  Then, given any
  continuous linear functional $l \in H^*$, there exist one and only one vector
  $y_l \in H$ such that
  \[
    l(x) = \langle x, y_l \rangle \qquad (\forall x \in H)
  \]
\end{thm}

\begin{proof}
  See Theorem 4.6-1 in \cite{ciarlet2013linear}.
\end{proof}

Hence, if $(\pi,H)$ is a finite dimensional representation of compact group $G$,
then all matrix coefficients of $H$ are of the from $g \mapsto L(\pi(g)v) =
\langle \pi(g)v, y_L \rangle$ for some $y_L \in H$.

\begin{lem}\label{schur-con-inter-op}
  Let $G$ be group, $(\pi_1, V_1)$ and $(\pi_2, V_2)$ be (complex)
  representations of $G$. Let $\langle , \rangle$ be any inner product on $V_1$.
  If $v_i,w_i \in V_i$, then the map $T : V_1 \to V_2$ defined by
  \[
    T(w) = \int_G \langle \pi_1(g)w,v_1 \rangle \cdot \pi_2(g^{-1}) v_2 dg
  \]
  is an interwining operator between $V_1$ and $V_2$.
\end{lem}

\begin{proof}
  We want to show, given $h \in G$,
  \[
    \pi_2(h)(T(v)) \stackrel{?}{=} T(\pi_1(h)(v))
  \]
  Let's expand the definition on the right hand side:
  \begin{align*}
    T(\pi_1(h)(v)) & = \int_G \langle \pi_1(g)(\pi_1(h)(v)),v_1 \rangle \cdot \pi_2(g^{-1}) v_2 dg\\
                   & = \int_G \langle \pi_1(gh)(v),v_1 \rangle \cdot \pi_2(g^{-1}) v_2 dg
  \end{align*}
  By change of variable $g \mapsto gh^{-1}$ gives:
  \begin{align*}
    T(\pi_1(h)(v)) & = \int_G \langle \pi_1(g)(v),v_1 \rangle \cdot \pi_2(hg^{-1}) v_2 d(gh^{-1})\\
                   & = \int_G \langle \pi_1(g)(v),v_1 \rangle \cdot \pi_2(h)(\pi_2(g^{-1}) v_2) d(gh^{-1})\\
                   & = \int_G \langle \pi_1(g)(v),v_1 \rangle \cdot \pi_2(h)(\pi_2(g^{-1}) v_2) dg && \text{(right invariance)}\\
                   & = \int_G  \pi_2(h) (\langle \pi_1(g)(v),v_1 \rangle \cdot (\pi_2(g^{-1}) v_2)) dg && \text{(linearity)}\\
                   & = \pi_2(h)(\int_G \langle \pi_1(g)v , v_1  \rangle \cdot \pi_2(g^{-1}) v_2 dg) && \text{(linearity)}\\
                   & = \pi_2(h)(T(v))
  \end{align*}
\end{proof}

\begin{thm}[Schur orthogonality, between representations] \label{schur-ortho-1}
  Let $G$ be a compact group, $(\pi_1, V_1)$, $(\pi_2, V_2)$ be two irreducible
  representations of $G$.  Then either $\mathcal{M}_{\pi_1} \perp
  \mathcal{M}_{\pi_2}$ in $L^2(G)$ or the representations are isomorphic.
\end{thm}

\begin{proof}
  We will prove the result by negating one of the conclusion and use it to prove
  the other.  Suppose $\mathcal{M}_{\pi_1} \not\perp \mathcal{M}_{\pi_2}$, then
  there exist matrix coefficients $f_1 \in \mathcal{M}_{\pi_1}$ and $f_2 \in
  \mathcal{M}_{\pi_2}$ such that $\langle f_1 , f_2 \rangle \neq 0$.  By Riesz
  representation theorem, we may characterize $f_1$ and $f_2$ by:
  \[
    f_1(g) = \langle \pi_1(g)w_1 , v_1 \rangle
    \quad
    f_2(g) = \langle \pi_2(g)w_2 , v_2 \rangle
  \]
  for some $w_i, v_i \in V_i$.  To avoid notational confusion, let
  $H(\cdot,\cdot)$ denote the inner product of the Hilbert space, and let
  $\langle \cdot, \cdot \rangle$ denote the inner product of both $V_1$ and
  $V_2$. Then by our assumption,
  \[
    H(f_1, f_2) = \int_G \langle \pi_1(g)w_1, v_1 \rangle \cdot
    \conj{\langle \pi_2(g)w_2, v_2 \rangle} dg \neq 0
  \]
  By complex conjugate, invariance and linearity,
  \begin{align*}
    & \int_G \langle \pi_1(g)w_1, v_1 \rangle \cdot \conj{\langle \pi_2(g)w_2, v_2 \rangle} dg\\
    & = \int_G \langle \pi_1(g)w_1, v_1 \rangle \cdot \langle v_2, \pi_2(g)w_2 \rangle dg\\
    & = \int_G \langle \pi_1(g)w_1, v_1 \rangle \cdot \langle \pi_2(g^{-1}) v_2, \pi_2(g^{-1}) \pi_2(g)w_2 \rangle dg\\
    & = \int_G \langle \pi_1(g)w_1, v_1 \rangle \cdot \langle \pi_2(g^{-1}) v_2, w_2 \rangle dg\\
    & = \int_G \langle \langle \pi_1(g)w_1, v_1 \rangle \cdot \pi_2(g^{-1}) v_2, w_2 \rangle dg\\
    & = \left\langle \int_G \langle \pi_1(g)w_1, v_1 \rangle \cdot \pi_2(g^{-1}) v_2 dg, w_2 \right\rangle \neq 0
  \end{align*}
  Let $T : V_1 \to V_2$ be define as in \Cref{schur-con-inter-op} and plugging
  the definition, we have,
  \[
    \langle T(w_1),w_2 \rangle \neq 0
  \]
  Hence $T : V_1 \to V_2$ is not zero, by Schur's Lemma (in compact case), $T$
  is an isomorphism.
\end{proof}

\Cref{schur-ortho-1} shows that if we have two non-isomorphic irreducible
representations of a compact group $G$, then any pair of matrix coefficients of
each representation is orthogonal.  We now consider the orthogonality of matrix
coefficients of the same irreducible representation, we will give an explicit
formula for computing the inner product of matrix coefficients of the same
irreducible representation.

\begin{thm}[Schur's Orthogonality, in one representation] \label{schur-ortho-2}
  Let $G$ be a compact group, $(\pi,V)$ be an irreducible representation of $G$.
  $V$ is equipped with inner product $\langle \cdot,\cdot \rangle$.
  Then fixing $v_1,v_2,w_1,w_2 \in V$, there exist a constant $d > 0$ such that
  \[
    \int_G \langle \pi(g)w_1, v_1 \rangle \cdot \conj{\langle \pi(g)w_2, v_2 \rangle} dg
    = d^{-1} \langle w_1, w_2 \rangle \cdot \langle v_2, v_1 \rangle
  \]
\end{thm}

\begin{proof}
  Firstly, let $v_1, v_2$ be fixed.  Let $T : V \to V$ be defined similar to
  \Cref{schur-con-inter-op} by:
  \[
    T(w) = \int_G \langle \pi(g)w, v_1 \rangle \pi(g^{-1})v_2 dg
  \]
  Then, by Schur's Lemma, $T = \lambda I$ for some constant $\lambda$ (depending
  on $v_1$ and $v_2$), and
  \begin{align*}
      & \int_G \langle \pi(g)w_1, v_1 \rangle \cdot \conj{\langle \pi(g)w_2, v_2 \rangle} dg\\
    = & \int_G \langle \pi(g)w_1, v_1 \rangle \cdot \langle v_2, \pi(g)w_2\rangle dg\\
    = & \int_G \langle \pi(g)w_1, v_1 \rangle \cdot \langle \pi(g^{-1})v_2, w_2\rangle dg\\
    = & \left\langle \int_G \langle \pi(g)w_1, v_1 \rangle \cdot \pi(g^{-1})v_2 dg , w_2 \right\rangle\\
    = & \langle T(w_1, w_2) \rangle = \lambda \langle w_1,w_2 \rangle
  \end{align*}
  Now, let $w_1, w_2$ be fixed. Let $\widetilde{T} : V \to V$ be defined by
  \[
    \widetilde{T}(v) = \int_G \langle \pi(g)v, w_2 \rangle \cdot \pi(g^{-1})w_1 dg
  \]
  Then, $\widetilde{T} = \widetilde{\lambda} I$, and
  \begin{align*}
      & \int_G \langle \pi(g)w_1, v_1 \rangle \cdot \conj{\langle \pi(g)w_2, v_2 \rangle} dg\\
    = & \int_G \langle \pi(g)w_1, v_1 \rangle \cdot \langle v_2, \pi(g)w_2 \rangle dg\\
    = & \int_G \langle \pi(g)w_1, v_1 \rangle \cdot \langle \pi(g^{-1})v_2, w_2 \rangle dg
  \end{align*}
  Substitute $g$ by $g^{-1}$, since Haar measures on compact groups are unique
  upto a constant multiple, let $c$ be the constant, then $d(g^{-1}) = cdg$,
  hence
  \begin{align*}
    \ldots = & c \int_G \langle \pi(g^{-1})w_1, v_1 \rangle \cdot \langle \pi(g)v_2, w_2 \rangle dg\\
    = & c \int_G \langle \langle \pi(g)v_2 w_2 \rangle \pi(g^{-1})w_1 , v_1 \rangle dg\\
    = & c \langle \int_G \langle \pi(g)v_2 w_2 \rangle \pi(g^{-1})w_1 dg , v_1 \rangle\\
    = & c \langle \widetilde{T}(v_2), v_1 \rangle = c \widetilde{\lambda} \langle v_2, v_1 \rangle
  \end{align*}
  Combining the two results above, there exist a constant $d$ such that
  \[
    \int_G \langle \pi(g)w_1, v_1 \rangle \cdot \conj{\langle \pi(g)w_2, v_2 \rangle} dg
    = d^{-1} \langle w_1, w_2 \rangle \cdot \langle v_2, v_1 \rangle
  \]
  If we let $w_1 = w_2$, $v_1 = v_2$, then the left hand side is positive the
  definition of inner product on the Hilbert space, then $d$ must be positive.
\end{proof}



% \subsection{Representation of Compact Groups}

% \begin{lem}[Complete reducibility]
%   If $G$ is compact, then each finite dimensional representation is the direct
%   sum of irreducible representations.
% \end{lem}

% \begin{proof}
  
% \end{proof}

% \begin{thm}[Schur orthogonality]
%   Suppose $(\pi_1, V_1)$ and $(\pi_2, V_2)$ are irreducible representations of a
%   compact group $G$. Either every matrix coefficient of $\pi_1$ is orthogonal in
%   $L^2(G)$ to every matrix coefficient coefficient of $\pi_2$, or the
%   representations are isomorphic.
% \end{thm}

% \begin{proof}
% \end{proof}

% \begin{defn}[Character, for compact group]

% \end{defn}


\subsection{Peter-Weyl's Theorem}

The goal of the this main section is to prove an arbitrary representation of a
compact group can be reduced into a direct sum of \emph{finite dimensional}
irreducible representations.  In particular, if we have the reducibility, by
\Cref{schur-ortho-1} and \Cref{schur-ortho-2}, we can show the components are
pairwise orthogonal.

The main technique used in the proofs follows from the idea of using matrix
coefficients to construct suitable finite dimensional subrepresentations of the
given representation.  Thus, what lies in the heart of Peter-Weyl's Theorem is
the observation that there exist an ``adequate'' supply of matrix coefficients
on the given group $G$.  Hence, we sometimes refer to this result
(\Cref{peter-weyl-1}) as the Peter-Weyl's Theorem.


Throughout the discussion of Peter-Weyl's Theorem, we will be mainly interested
in normed vector space of continuous functions on $G$, or space of $L^p$
functions on $G$, where the $p$-norm is defined as, for $1 \leq p < \infty$
\[
  \norm{f}_p \triangleq \left\{ \int_G \abs{f(g)}^{p}   \right\}^{1/p}
\]
when $p = \infty$, we define the sup-norm by
\[
  \norm{f}_{\infty} \triangleq \esssup_G f = \inf \SetForm{a \in \R}{\mu\SetForm{g \in G}{f(x) > a} = 0}
  = \sup_{G} \abs{f(x)} \quad \text{if } f \in C(G)
\]
We define $L^p(G)$ to be the set containing all functions $f$ on $G$ such that
$\norm{f}_p < \infty$.

Let $C(G)$ denote the collection of continuous functions on compact group $G$,
then $C(G)$ forms a ring with multiplication being \emph{convolution}:
\[
  (f_1 \star f_2)(g) = \int_G f_1(gh^{-1}) f_2(h) dh = \int_G f_1(h) f_2(h^{-1}g) dh
\]

Given $\phi \in C(G)$, let $T_{\phi} : C(G) \to C(G)$ be the linear operator by
left convolution: $T_{\phi} : f \mapsto \phi \star f$.

\begin{prop} \label{1-2-infty}
  Let $f \in C(G)$ be given, then
  \[
    \norm{f}_1 \leq \norm{f}_2 \leq \norm{f}_{\infty}
  \]
\end{prop}

\begin{proof}
  By Cauchy-Schwarz inequality, let $\langle \cdot , \cdot \rangle$ be the inner
  product defined as \Cref{Hilbert-inner-product}, let $1$ denote the constant
  function $1$,
  \[
    \norm{f}_1 = \langle \abs{f} , 1 \rangle
    \leq \left(\langle \abs{f}, \abs{f} \rangle \right)^{1/2} \cdot \left( \langle 1, 1 \rangle \right)^{1/2}
    = \left( \int_G \abs{f(g)}^2 dg \right)^{1/2} = \norm{f}_2
  \]
  The second inequality is trivial,
  \[
    \norm{f}_2
    = \left( \int_G \abs{f(g)}^2 dg \right)^{1/2}
    \leq \left( \int_G \norm{f}_{\infty}^{2} dg \right)^{1/2}
    = \norm{f}_{\infty}
  \]
\end{proof}

\begin{prop} \label{Tphi-bounded}
  Let $G$ be a compact group, $\phi \in C(G)$.  Then $T_{\phi}$ is a bounded
  operator on $L^1(G)$. Further, if $f \in L^1(G)$, then $T_{\phi}f \in
  L^{\infty}(G)$ and
  \[
    \norm{T_{\phi}f}_{\infty} \leq \norm{\phi}_{\infty} \norm{f}_1
  \]
\end{prop}

\begin{proof}
  Let $f \in L^1(G)$ be given. We estimate the sup-norm of $T_{\phi}f$ by
  \[
    \norm{T_{\phi} f}_{\infty}
    = \sup_{g \in G} \abs{\int_G \phi(g h^{-1}) f(h) dh}
    \leq \norm{\phi}_{\infty} \int_G \abs{f(h)} dh
  \]
  Put $C \triangleq \norm{\phi}_{\infty}$, $C < \infty$ since $\phi$ is a
  continuous function on a compact set hence obtain a finite extrema.  Since $f
  \in L^1(G)$, $\norm{f}_1 < \infty$, thus $T_{\phi}f \in L^{\infty}(G)$.
  
  Furthermore by \Cref{1-2-infty}, $\norm{T_{\phi} f}_1 \leq \norm{T_{\phi} f}_2
  \leq \norm{T_{\phi} f}_{\infty}$,  We have, $\norm{T_{\phi}f}_1 \leq C
  \norm{f}_1$, thus by definition the operator $T_{\phi}$ is bounded in
  $L^1(G)$. In fact, $T_{\phi}$ is bounded in all three norms: $1,2,\infty$.
\end{proof}

\begin{prop}\label{self-adjoint-compact}
  Let $G$ be a compact group, $\phi \in C(G)$, then
  \begin{enumerate}
    \item $T_{\phi}$ is a bounded operator on $L^2(G)$ and $\abs{T_{\phi}} \leq \supnorm{\phi}$.
    \item $T_{\phi}$ is compact.
    \item If $\phi(g^{-1}) = \conj{\phi(g)}$, it is self-adjoint.
  \end{enumerate}
\end{prop}

\begin{proof}
  \begin{enumerate}
    \item Let $f \in L^2(G)$ be given, then $f \in L^1(G)$ by \Cref{1-2-infty}
      and by the argument in \Cref{Tphi-bounded},
      \[
        \norm{T_{\phi} f}_{2} \leq
        \supnorm{T_{\phi} f} \leq
        \supnorm{\phi} \norm{f}_1 \leq \supnorm{\phi} \norm{f}_2
      \]
      Hence $T_{\phi}$ is bounded on $L^2(G)$ and $\abs{T_{\phi}} \leq \supnorm{\phi}$.

    \item 
      Let $B$ be a set of bounded functions in $L^2(G)$, without loss of
      generality, we consider the unit ball in $L^2(G)$. Since $L^2(G) \subset
      L^1(G)$, it suffices to consider the unit ball in $L^1(G)$ , that is:
      \[
        B \triangleq \SetForm{f \in L^1(G)}{\norm{f}_1 \leq 1}
      \]
      We want to show the image set $T_{\phi}(B)$ is sequentially compact, that
      is, every infinite sequence in $T_{\phi}(B)$ has a convergent subsequence.
      We will establish the result by using Ascoli and Arzela
      \Cref{ascoli-arzela}.  First off, by \Cref{Tphi-bounded}, we know
      $T_{\phi}(B)$ is bounded, hence it suffice to show $T_{\phi}(B)$ is
      equicontinuous.

      Let $\epsilon > 0$ be given, since $\phi \in C(G)$ and $G$ is compact,
      $\phi$ is uniformly continuous, there exist a neighborhood $N$ of the
      identity $1_G \in G$ such that
      \[
        \abs{\phi(kg) - \phi(g)} < \epsilon \qquad (\forall k \in N)
      \]
      Then given $f \in B$, $k \in N$, $g \in G$,
      \begin{align*}
        \abs{T_{\phi}f(kg) - T_{\phi}f(g)}
        & = \abs{\int_G (\phi(kgh^{-1}) - \phi(gh^{-1})) f(h) dh}\\
        & \leq \int_G \abs{\phi(kgh^{-1}) - \phi(gh^{-1})} \cdot \abs{f(h)} dh\\
        & \leq \epsilon \norm{f}_1 \leq \epsilon
      \end{align*}
      By definition, $T_{\phi}(B)$ is equicontinuous, hence apply
      \Cref{ascoli-arzela}, $T_{\phi}(B)$ is sequentially compact, $T_{\phi}$ is
      compact.

    \item Suppose we have $\phi(g^{-1}) = \conj{\phi(g)}$, then
      \begin{align*}
        \langle T_{\phi}f_1, f_2 \rangle
        & = \int_G T_{\phi}f_1(g) \cdot \conj{f_2(g)} dg\\
        & = \int_G (\int_G \phi(gh^{-1}) f_1(h) dh) \cdot \conj{f_2(g)} dg\\
        & = \int_G (\int_G \conj{\phi(hg^{-1})} f_1(h) dh) \cdot \conj{f_2(g)} dg\\
        & = \int_G f_1(h) \cdot \conj{T_{\phi}f_2(h)} dh\\
        & = \langle f_1, T_{\phi}f_2 \rangle
      \end{align*}
      Hence $T_{\phi}$ is self-adjoint.
  \end{enumerate}
\end{proof}

\begin{prop} \label{Tphi-invariant}
  Let $G$ be a compact group, $\phi \in C(G)$, $\lambda \in \C$, then the
  $\lambda$-eigenspace
  \[
    V(\lambda) = \SetForm{f \in L^2(G)}{T_{\phi}f = \lambda f}
  \]
  is invariant under $\rho(g)$ for all $g \in G$, where $\rho(g)$ is defined as
  usual as in \Cref{eq:LR-translation}.
\end{prop}

\begin{proof}
  Let $g \in G$ be given,
  \begin{align*}
    (T_{\phi}(\rho(g)f))(x)
    & = \int_G \phi(xh^{-1}) f(hg) dh\\
    & = \int_G \phi(xgh^{-1}) f(h) dh && \text{substitute } h \to hg^{-1}\\
    & = (\rho(g)(T_{\phi}f))(x)
  \end{align*}
\end{proof}

\begin{thm}[Peter-Weyl, Part 1] \label{peter-weyl-1}
  The matrix coefficients of $G$ are dense in $L^2(G)$.
\end{thm}

\begin{proof}
  Let $f \in L^2(G)$ be given, we need to show for any $\epsilon > 0$, there
  exist a matrix coefficient $f'$ (thus associated with a finite dimensional
  representation), such that $\norm{f' - f} < \epsilon$.

  Let $\epsilon > 0$ be given, since $G$ is compact, $f$ is uniformly continuous
  on $G$, hence left translation of $f$ by $g$: $\lambda(g)f = h \mapsto
  f(g^{-1}h)$ is uniformly continuous on $G$. Then there exist a neighborhood $U
  \ni 1_G$ such that $\norm{\lambda(g)f - \lambda(1_G)f}_{\infty} =
  \norm{\lambda(g)f - f}_{\infty} < \epsilon/2$ for all $g \in U$.

  Define $\phi : U \to \C$ be a continuous map satisfying $\int_G \phi(g) dg = \int_U
  \phi(g) dg = 1$, and $\phi(g) = \phi(g^{-1})$.  Then the operator $T_{\phi}$
  is self-adjoint and compact by \Cref{self-adjoint-compact}. We collect the
  following facts:
  \begin{enumerate}
    \item $\norm{T_{\phi}f - f} < \epsilon/2$:
      \begin{align*}
        \abs{T_{\phi}f(h) - f(h)}
        & = \abs{\int_G \phi(g) f(g^{-1}h) - \phi(g)f(h) dg}\\
        & = \abs{\int_U \phi(g) f(g^{-1}h) - \phi(g)f(h) dg}\\
        & \leq \int_U \phi(g) \cdot \abs{f(g^{-1}h) - f(h)} dg\\
        & \leq \int_U \phi(g) \cdot \norm{\lambda(g)f - f}_{\infty} dg
          \leq \frac{\epsilon}{2}
      \end{align*}

    \item $T_{\phi}$ is a compact operator on $L^2(G)$ by \Cref{self-adjoint-compact}.

    \item 
      Let $\lambda$ denote an eigenvalue of $T_{\phi}$, by spectral theorem
      \Cref{spectral}, the eigenspaces $V(\lambda) \subset L^2(G)$ are all
      finite dimensional except perhaps $V(0)$.  Furthermore, the spectral
      theorem tells us spaces $V(\lambda)$ are mutually orthogonal and they span
      $L^2(G)$.

    \item 
      By \Cref{Tphi-invariant}, maps in $V(\lambda)$ are all $T_{\phi}$
      invariant.
  \end{enumerate}

  Let $f_\lambda$ be the projection of $f$ on the basis $V(\lambda)$:
  \[
    f_{\lambda} = \sum_{w \in V(\lambda)} \langle f, w \rangle w
  \]
  By orthogonality between $V(\lambda)$s, we have
  \[
    \sum_{\lambda} \norm{f_{\lambda}}_2^2 = \norm{f}_2^2 < \infty
  \]
  For some positive real number $q$, define $f', f''$ by:
  \[
    f'' \triangleq \sum_{\abs{\lambda} > q} f_{\lambda},
    \qquad
    f' \triangleq T_{\phi}f''
  \]
  Then both $f'$ and $f''$ are contained in a finite dimensional vector space:
  $\bigoplus_{\abs{\lambda} > q} V(\lambda)$.  By (4), and apply
  \Cref{left-right-finite}, it follows $f$ is a matrix coefficient of a finite
  dimensional representation.

  Lastly, we will show $f'$ indeed approximates $f$ within $\epsilon$: choose
  $q$ so that
  \[
    \norm{\sum_{0 < \abs{\lambda} < q} f_{\lambda}}_1
    \leq \norm{\sum_{0 < \abs{\lambda} < q} f_{\lambda}}_2
    = \sqrt{\sum_{0 < \abs{\lambda} < q} \norm{f_{\lambda}}_2^2}
    < \frac{\epsilon}{2\supnorm{\phi}}
  \]
  Then,
  \begin{align*}
    \supnorm{T_{\phi}(f - f'')}
    & = \supnorm{T_{\phi} \left( f_0 + \sum_{0 < \abs{\lambda} < q} f_{\lambda} \right)}
    = \supnorm{T_{\phi} \left( \sum_{0 < \abs{\lambda} < q} f_{\lambda} \right)}\\
    & \leq \supnorm{\phi} \cdot \norm{\sum_{0 < \abs{\lambda} < q} f_{\lambda}}_1
    < \frac{\epsilon}{2}
  \end{align*}
  Therefore, use $T_{\phi}$ invariance,
  \[
    \supnorm{f - f'}
    = \supnorm{f + T_{\phi}(f - f - f'')}
    = \supnorm{f - T_{\phi}f} + \supnorm{T_{\phi}f - T_{\phi}f''}
    < \frac{\epsilon}{2} + \frac{\epsilon}{2} = \epsilon
  \]
\end{proof}

\begin{cor}
  The matrix coefficients of compact group $G$ are dense in $L^2(G)$.
\end{cor}

\begin{proof}
  Since $C(G)$ is dense in $L^2(G)$, it follows from \Cref{peter-weyl-1}.
\end{proof}

% \begin{thm}
% \end{thm}

% \begin{proof}
% \end{proof}

\begin{thm}[Peter-Weyl, Part 2] \label{peter-weyl-2}
  Let $H$ be a Hilbert space and $G$ be a compact group. Let $\pi : G \to
  End(H)$ be a unitary representation. Then $H$ is a direct sum of
  finite dimensional irreducible representations.
  % The space of square integrable functions on $G$ is the Hilbert space direct
  % sum over finite dimensional irreducible representations $V$,
  % \[
  %   \ell^2(G) \cong \hat{\bigoplus} End(V)
  % \]
\end{thm}

\begin{proof}
  We will prove firstly, if $H$ is nonzero then it has an irreducible finite
  dimensional invariant subspace. Then, we will extract a finite dimensional
  representation of $G$ on $H$.

  Let $v \in H$ be a nonzero vector, since the group action $\pi$ is continuous
  and $G$ is compact, it's uniformly continuous in $G$. Hence there exist a
  neighborhood $N$ of $1_G \in G$ such that
  \[
    \norm{\pi(g)v - \pi(1_G)v} = \norm{\pi(g)v - v} \leq \frac{\norm{v}}{2} \qquad (\forall g \in N)
  \]
  Define $\phi : N \to \C$ be a continuous map satisfying $\int_G \phi(g) dg =
  1$. Then
  \[
    \left\langle \int_G \phi(g) \cdot \pi(g)v dg , v \right\rangle
    = \langle v, v \rangle - \left\langle \int_N \phi(g) \cdot (v - \pi(g)v) dg, v \right\rangle
  \]
  and by Cauchy Schwarz,
  \begin{align*}
    & \norm{ \left\langle \int_N \phi(g) \cdot (v - \pi(g)v) dg, v  \right\rangle }\\
    & \leq \norm{\int_N \phi(g) \cdot (v - \pi(g)v) dg} \cdot \norm{v}\\
    & \leq \int_N \norm{\phi(g)} \norm{v - \pi(g)v} dg \cdot \norm{v}\\
    & \leq \int_N \norm{\phi(g)} \cdot \frac{\norm{v}}{2} dg \cdot \norm{v}
      = \frac{\norm{v}^2}{2}
  \end{align*}
  Hence, the integral $\int_G \phi(g) \cdot \pi(g)v dg \neq 0$.

  By \Cref{peter-weyl-1}, since $\phi \in C(G)$, for arbitrary $\epsilon > 0$,
  there exist a matrix coefficient $f$ associated to a finite dimensional
  representation $(\rho, W)$ such that $\supnorm{f - \phi} < \epsilon$. Thus, in
  this case,
  \[
    \norm{\int_G (f - \phi)(g) \cdot \pi(g)v dg} \leq \epsilon \norm{v}
  \]
  Take small enough $\epsilon$, then $\norm{\int_G f(g) \cdot \pi(g)v dg} =
  \norm{\int_G \phi(g) \cdot \pi(g)v dg} + \epsilon \norm{v} \neq 0$.

  By \Cref{dual-mat-coeff}, $\check{f}$ is a matrix coefficient of the dual
  representation $(\hat{\rho}, W^*)$ (which has the same dimension). By
  definition of matrix coefficient, $\check{f}(g) = L(\hat{\rho}(g)w)$ for some $L \in
  (W^* \to \C) \simeq W$, $w \in W^*$. Define $T : W^* \to H$ by
  \[
    T(x) \triangleq \int_G L(\hat{\rho}(g^{-1})x) \cdot \pi(g)v dg
  \]
  using the same argument used to prove \Cref{schur-con-inter-op}, we know $T
  \in \Hom^G(W^*, H)$ and it's nonzero since when evaluated at $w$:
  \[
    T(w) = \int_G L(\hat{\rho}(g^{-1})w) \cdot \pi(g)v dg
    = \int_G \check{f}(g^{-1}) \cdot \pi(g)v dg
    = \int_G f(g) \cdot \pi(g)v dg \neq 0
  \]
  Since $\dim(W^*) < \infty$, the image $T(W^*)$ is a nonzero finite dimensional
  invariant subspace of $H$!

  Now, let $\Sigma$ be the set of all sets of finite dimensional irreducible
  invariant subspaces, order its elements by inclusion.  By Zorn's Lemma, the
  partial order $\Sigma$ has a maximal element $\widetilde{S}$.  If
  $\widetilde{S}$ spans $H$, we are done with $\widetilde{S}$ being a complete
  decomposition of $H$.  Otherwise, the complement of the span of $S$ is a
  nonzero, by previous arguments, it contains an invariant irreducible subspace
  $U$, then $\widetilde{S}$ is not maximal since $\widetilde{S} \cup \{U\}$ is
  larger than $\widetilde{S}$, we have a contradiction!
\end{proof}

We have shown that a Hilbert space representation of a compact group may be
completely reduced into a direct sum of \emph{finite dimensional} irreducible
representations.  If we instantiate the result with $G = \T$, $H = L^2(\R)$ and
an action $\widetilde{\pi} : \T \times L^2(\R) \to L^2(\R)$
\[
  \widetilde{\pi}(e^{i\theta})f \triangleq e^{i\theta} \cdot \hat{f}
\]
where $\hat{f}$ is the complete decomposition of $f$ into orthogonal components
given by Peter-Weyl's Theorem (which is exactly the Fourier transform of $f$).
If we chain $\widetilde{\pi}$ together with the quotient map $\R \to \R/\Z
\simeq \T$ that sends $t \in \R$ to $e^{it} \in \T$, then we recover the map
$\pi : \R \times L^2(\R) \to L^2(\R)$ from the introduction.

% \newpage
% \input{compact-abel-group.tex}

\section{Conclusion}

The theory Fourier of analysis was long developed throughout the history: it was
first developed in $16$th century to study the physical phenomenon of vibration
of strings and to solve partial differential equations associated with it.  It
later became one of the core problems in analysis and relates to varies areas of
mathematics.  For example, it has relates intimately to the representation
theory on compact groups as we have explored in this paper.

In fact, if we put more restriction on the case we studied and let $G$ be a
compact \emph{abelian} group (of which $\T$ still is one archetypal example),
there is an even more remarkable symmetry between representation theory and
Fourier analysis: characters of $G$ form an orthonormal basis of $L^2(G)$ and
Fourier Transform and Fourier inversion formulas gives an isomorphism between
$G$ and the dual group $\hat{G}$ of its characters.

% Fourier analysis was studied extensively in the $20$th century and
% remains a


% \section{Categorified Fourier Analysis}

% \begin{thm}
%   The category of representations of group $G$ is equivalent to the category of
%   vector spaces parametrized by the dual group $\hat{G}$ of characters.
% \end{thm}


\newpage
\bibliographystyle{amsplain}
\nocite{*}
\bibliography{refs}

\end{document}
