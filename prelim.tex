\section{Preliminaries}

Let $G$ be a group, that is, an underlying set $\mathbb{G}$ with operations that
give $\mathbb{G}$ the group structure, if we further enforce a topological
structure on $\mathbb{G}$, then we can treat it as a topological space given
those algebraic operations are continuous. We call $G$ a \emph{topological group}.

Some examples of topological groups include the group $\R^{\times}$ of real
numbers under multiplication: its' topology was induced from the metric $d(x,y)
= \abs{x - y}$.  Since the topology of $\R^{\times}$ is the usual one induced by
the metric, by $\epsilon-\delta$ argument, it's easy to show the map $(x,y)
\mapsto x \times y$ from $\R^2$ to $\R$ is continuous.

Another example of a topological group that is closely related to Fourier
analysis is the circle group $\T = \SetForm{z \in \C}{\abs{z} = 1}$, under
multiplication.  Equivalently, we can represent elements of $\T$ in polar
coordinates: $\T = \SetForm{e^{i\theta}}{\theta \in [-\pi,\pi]}$.  The
multiplicative identity is $e^{0} = 1$, and inverse operations is simply
$\left(e^{i\theta} \right)^{-1} = e^{-i\theta}$.

Apparently, $\T$ is not a finite group since the interval $[-\pi, \pi]$ is
uncountable, it's a topological group where the topological structure is
inherited from the usual topological structure of $\C$. Furthermore, $\T$ is a
compact group since we know the unit circle is compact in $\C$.

\subsection{Measure Theory Backgrounds}

Let $X$ be a topological space, we say $X$ is a \emph{measure space} if there
exist a $\sigma$-algebra $\mathfrak{M}$ in $X$, that is, a family of subsets of
$X$ containing $X$ and is closed under taking complements and countable union.
We call elements of $\mathfrak{M}$ \emph{measurable sets}. Let $f : X \to Y$ be
a map from a measurable space $X$ to a topological space $Y$ \emph{measurable}
if for every open set $V$ in $Y$, the pullback $f^{-1}(V)$ is a measurable set.

Let $\mathcal{T}$ be a topology on $X$, let $\Omega$ be a family of all
$\sigma$-algebras in $X$ that contains $\mathcal{T}$.  Then by intersecting all
elements in this family, we have a \emph{smallest} $\sigma$-algebra
$\mathcal{B}$ that contains all open sets of $X$, we refer to elements of
$\mathcal{B}$ as \emph{Borel sets}.

Given $(X,\mathfrak{M})$ a measure space, we define a \emph{measure} to be a map
$\mu : \mathfrak{M} \to [0,\infty]$ that is countably additive. This means, for
any disjoint countable collection $\{A_j\} \subset \mathfrak{M}$,
\[
  \mu\left( \bigcup_{j = 0}^{\infty} A_j \right) = \sum_{j = 0}^{\infty} \mu(A_j)
\]
If the $\sigma$-algebra above is the collection of Borel sets of $X$, then we
call the measure a \emph{Borel} measure.  We say the Borel measure is
\emph{regular} if it satisfies, for all measurable set $X$,
\[
  \mu(X) = \inf \SetForm{\mu(U)}{U \supset X, U \text{ open}} = \sup
  \SetForm{\mu(K)}{K \subset X, K \text{ compact}}
\]
If the topological space $G$ is a \emph{locally compact} group (i.e. every point
of $G$ has a compact neighborhood), there is, up to a constant multiple, a
unique regular Borel measure $\mu_L$ that is invariant under left translation.
By \emph{left translation invariance}, we mean $\mu_L(X) = \mu_L(gX)$ for all
measurable sets $X$ and $g \in G$.  We call such measure a \emph{left Haar
  measure}, it should also satisfy $\mu_L(X) < \infty$ for $X$ compact and
$\mu_L(X) > 0$ for all $X$ measurable. (For the existence and uniqueness of
$\mu_L$, see \cite{halmos2013measure}).  Similarly, we can also define
\emph{right Haar measure}.  Usually left and right Haar measure do not coincide
for general topological groups, when they do, we call the group
\emph{unimodular}.

One of the main purpose of inventing the formalism of measure is to develop the
theory of integration with respect to a given measure.  The idea of computing
the integral $\int_X f(x) d\mu(x)$ is to approximate the function $f$ by
\emph{simple functions}, that is, a linear combinations of characteristic
function on measurable sets.  If $f : X \to \R$ is non-negative, define
\[
  \int_X f(x) d\mu(x) \triangleq \sup_{0 \leq s \leq f} \int_X s(x) d\mu(s)
  = \sup_{0 \leq \sum \lambda_i \chi_{E_i} \leq f} \int_X \sum_{i = 0}^{n} \lambda_i \chi_{E_i}(x) d\mu(x)
\]
% Similar to the finite case, a representation of $G$ is a vector space $V$
% associated with a group action $\pi$.  If $V$ is finite-dimensional, then the
% action $\pi : G \to GL(V)$ is a continuous homomorphism where the topology of
% $GL(V)$ is inherited from the space $End(V)$ of linear self-maps.  In this case,
% we can define \emph{characters} similarly as a complex valued function
% $\chi_{\pi}(g) = \Tr \pi(g)$, we say a character is \emph{irreducible} if it's a
% character of a irreducible representation (i.e. contains no proper nonzero invariant spaces).

% On the other hand, if $V$ is infinite dimensional, we instead say $V$ is
% irreducible if it has no proper nonzero invariant \emph{closed} subspaces.

Notice, if $\mu_L$ is a left Haar measure, then for $\gamma \in G$,
\[
  \int_G f(\gamma g) d \mu_L(g) = \int_G f(\gamma g) d\mu_L(\gamma g)
\]
substituting $h = \gamma g$, we have
\[
  \int_G f(\gamma g) d \mu_L(g) = \int_G f(g) d\mu_L(g)
\]

Similarly, to check $\mu_R$ is a right Haar measure, it amounts to check the
integral
\[
  \int_G f(g \gamma) d \mu_L(g) \stackrel{?}{=} \int_G f(g) d \mu_L(g)
\]

Consider the conjugation map $\phi(g) : h \mapsto g^{-1} h g$, it's a
automorphism on $G$ (with inverse being $\phi(g)^{-1} : h \mapsto g h g^{-1}$).
Since left Haar measures are unique up to a constant multiple, the left Haar
measure of $\phi_g(G)$ is a constant multiple of the left Haar measure of $G$.
In another words, there exist a function $\delta : G \to \R_+^{\times}$ such that
\[
  \int_G f(\phi_g(h)) d\mu_L(h) = \delta(g) \int_G f(h) d\mu_L(h)
\]

Given $g_1, g_2 \in G$, since conjugating first by $g_1$ then by $g_2$ is
equivalent to conjugating by $(g_2g_1)$, we have $\delta(g_2) \cdot \delta(g_1)
= \delta(g_2g_1)$, the map $\delta$ is a group homomorphism.
% Generally, we call a homomorphism from $G$ to $\C^{\times}$ a \emph{quasicharacter}.
% Then composing
% it in the integral $\int_G f(\phi g) d\mu_L(g)$ should given a constant multiple
% of $\int_G f(g) d\mu_L(g)$.

\begin{lem}
  Let $\mu_L$ be a left Haar measure of $G$, let $\delta : G \to \R_+^{\times}$
  be defined as above, then $\delta \cdot \mu_L$ is a right Haar measure of $G$.
\end{lem}

\begin{proof}
  We verify the measure $\delta(h)\mu_L(h)$ is right invariant.  Let
  $\widetilde{f} = f \cdot \delta$, then by definition of $\delta$,
  \[
    \int_G \widetilde{f}(g^{-1} h g) d\mu_L(h) =
    \delta(g) \int_G \widetilde{f}(h) d\mu_L(h)
  \]
  Since $\mu_L$ is left invariant, left translate $\widetilde{f}$ by $g$ gives
  \[
    \int_G \widetilde{f}(hg) d\mu_L(h) = \delta(g) \int_G \widetilde{f}(h) d\mu_L(h)
  \]
  Since $\delta$ is a homomorphism, we expand the left hand side:
  \[
    \int_G f(hg) \delta(h) \delta(g) d\mu_L(h) = \delta(g) \int_G \widetilde{f}(h) d\mu_L(h)
  \]
  Canceling $\delta(g)$ from both sides, we have
  \[
    \int_G f(hg) \delta(h) d\mu_L(h) = \int_G \widetilde{f}(h) d\mu_L(h) =
    \int_G f(h) \delta(h) d\mu_L(h)
  \]
  Thus, $\delta \cdot \mu_L$ is a right Haar measure.
\end{proof}

The property of Haar measure on compact groups follows from the Lemma above:

\begin{lem}
  Let $G$ be a compact group, then
  \begin{enumerate}
    \item[(1)] $G$ is unimodular
    \item[(2)] $\mu_L(G) < \infty$
  \end{enumerate}
\end{lem}

\begin{proof}
  Let $\delta : G \to \R_+^{\times}$ be define as above.  Since $G$ is compact
  and $\delta$ is continuous, the image $\Ima(\delta)$ is a compact subgroup of
  $\R_+^{\times}$.  Since the only compact subgroup of $\R_+^{\times}$ is
  trivial, the image is trivial.  Then the right Haar measure $\delta \cdot
  \mu_L$ is equal to $\mu_L$, hence $G$ is unimodular.  Since $G$ is compact, by
  assumption for Haar measure, $\mu(G) < \infty$.
\end{proof}

Since $\mu_L(G) < \infty$ and left Haar measures (also right Haar measure when
$G$ is compact) are unique upto a constant multiple, we can pick a measure $\mu$
such that the integral is normalized, that is, $\int_G 1 d\mu(g) = 1$. Without
loss of generality, we will always consider Haar measures that normalizes the
integral in later sections. For notational convenience, later in the paper we
will use $dg$ in place for $d\mu(g)$.

\subsection{Functional Analysis Machineries}

In order to discuss whether we have ``enough'' matrix coefficients, we will
first need to develop some results on linear operators, in particular
on \emph{compact operators}.

Let $\mathfrak{H}$ be a normed vector space with norm $\norm{-}$, a linear
operator $T : \mathfrak{H} \to \mathfrak{H}$ (i.e. maps compatible with the
linear structure of $\mathfrak{H}$) is called \emph{bounded} if there exist a
constant $C$ such that
\[
  \norm{Tx} \leq C\norm{x} \qquad (\forall x \in \mathfrak{H})
\]
The smallest such constant is referred to as the \emph{operator norm} of $T$,
denote by $\abs{T}$.  It's easy to see $T$ is bounded iff $T$ is continuous (on
the metric topology given by the norm).

We call the operator $T$ \emph{self-adjoint} if
\[
  \langle T f, g \rangle = \langle f, T g \rangle \qquad (\forall f,g \in \mathfrak{H})
\]

A bounded operator $T : \mathfrak{H} \to \mathfrak{H}$ is \emph{compact} if
given any bounded sequence $\{x_1, x_2, \ldots\} \subset \mathfrak{H}$, the
image sequence $\{Tx_1, Tx_2, \ldots\}$ has a convergent subsequence.

For $f,g \in \mathfrak{H}$, we call $f$ an \emph{eigenvector} with
\emph{eigenvalue} $\lambda$ of $T$ if $f \neq 0$ and $Tf = \lambda f$. As usual,
we call the space $\SetForm{f \in \mathfrak{H}}{f \neq 0, Tf = \lambda f} \cup \{0\}$ the
$\lambda$-eigenspace.

\begin{thm}[Spectral theorem for compact operators] \label{spectral}
  Let $T$ be a compact self-adjoint operator on a Hilbert space $\mathfrak{H}$.
  Let $\mathfrak{N}$ be the nullspace of $T$. Then
  \begin{enumerate}
    \item[(1)] the Hilbert space dimension of $\mathfrak{N}^{\perp}$ is at most countable.
    \item[(2)] $\mathfrak{N}^{\perp}$  has an orthonormal basis
      $\{\phi_i\}_1^{\infty}$ of eigenvectors of $T$, and associated set of
      eigenvalues $\{\lambda_i\}_1^{\infty}$.
    \item[(3)] If $\mathfrak{N}^{\perp}$ is not finite-dimensional, the
      eigenvalues $\lambda_i \to 0$ as $i \to \infty$.
  \end{enumerate}
\end{thm}

\begin{proof}
  See Theorem 4.11-1 in \cite{ciarlet2013linear}.
\end{proof}

Notice, if $\mathfrak{N}^{\perp}$ is not finite-dimensional, then each
$\lambda_i$ may only appear finitely many times in the sequence of eigenvalues,
hence there are only finitely many eigenvectors associated with $\lambda_i$, the
$\lambda_i$-eigenspace is finite.

Let $X$, $Y$ be compact topological spaces, $Y$ is has a metric $d$.  Let $U$ be
a set of continuous maps $X \to Y$, we call $U$ being \emph{equicontinuous} if
for every $x \in X$ and $\epsilon > 0$, there exists a neighborhood $N \ni x$
such that
\[
  d(f(x), f(x')) < \epsilon \qquad (\forall x' \in N, f \in U)
\]

The following theorem relates equicontinuity with compactness:

\begin{thm}[Ascoli and Arzela] \label{ascoli-arzela}
  Let $X$ be a compact space, $U \subset C(X)$ is a bounded subset of continuous
  functions, with the sup-norm metric. Suppose $U$ is equicontinuous, then every
  sequence in $U$ has a uniformly convergent subsequence.
\end{thm}

\begin{proof}
  See Theorem 3.10-1 in \cite{ciarlet2013linear}.
\end{proof}